Neste trabalho foi desenvolvido um sistema SLAM multiagente distribuído e 
descentralizado, no qual todo o processamento é feito pelos agentes sem a presença de uma figura reguladora central, utilizando o Filtro de Informação Estendido Esparso e explorando seu baixo uso de memória e tempo de processamento constante (desconsiderando-se associação de dados). Para seu desenvolvimento utilizou-se o \textit{framework} ROS e o 
simulador Gazebo.

Essas duas ferramentas contam com ampla base de usuários e portanto há 
muito conteúdo em comunidades e fóruns que pode ser consultado ao longo 
do desenvolvimento para sanar dúvidas de integração e funcionamento. 
Além disso, como são ferramentas de código livre, podem ser utilizadas 
sem a necessidade de adquirir licenças; esse fato, somado à 
disponibilização integral da implementação realizada neste trabalho\footnote{Implementação disponível em: \url{https://github.com/wvmcastro/slam}}, potencializa a 
possibilidade de reuso e melhoramento deste por outros. 

Outras contribuições são as deduções e ilustrações que 
foram produzidas para preencher lacunas dos livros e artigos 
consultados ao longo do desenvolvimento deste trabalho. Sendo as mais importantes: (I) A 
Seção \ref{sec:ekf-slam-landmark-insertion} que apresenta dedução da 
covariância de uma nova \textit{landmark} no EKF-SLAM. (II) A Figura 
\ref{fig:seif-slam-prediction-variables}, que ilustra a esparsidade das 
matrizes intermediárias utilizadas na etapa de predição do SEIF-SLAM. 
(III) A Seção \ref{fig:seif-slam-landmark-insertion} análoga ao que 
foi feito para o EKF-SLAM, traz a dedução da informação de uma nova 
\textit{landmark} no SEIF-SLAM. (IV) A combinação da sequência de passos 
no processamento da leitura do sensor LiDAR, apresentada na Seção \ref{sec:lidar-data-processing}. (V) Errata da equação de transformação de 
sistema de referência do vetor e matriz de informação do SEIF 
apresentada em \cite[Seção.~12.11.2]{thrun2005probabilistic}, e aqui 
deduzida na Equação \ref{eq:seif-reference-transform}.


\section{Trabalhos futuros}
\begin{itemize}
  \item \textbf{Investigação de resultados}: É preciso confirmar, através da 
  repetição sistemática de simulações, a degradação no desempenho de estimação do SEIF ao aumentar a cardinalidade do conjunto de \textit{landmarks} ativas, apresentada na Seção \ref{sec:exp-estimate-comparison}. Caso confirmado, é necessário revisar a 
  implementação para procurar possíveis erros já que esse resultado 
  destoa da literatura e compreender quais circunstâncias causam esse efeito.
  \item \textbf{Coordenação entre agentes}: A falta de 
coordenação entre os agentes limitou o ganho de eficiência esperado de 
um sistema com mais de um robô, como discutido na Seção \ref{sec:exp-known-initial-pose}. Dessa forma, um trabalho futuro deve focar na coordenação entre agentes com o objetivo de dividir a carga de trabalho.
  \item \textbf{Troca de mapas}: Em relação à troca de mapas, há espaço para melhorias na técnica de registro de nuvem de pontos utilizada, possibilitando maior taxa de sucesso na troca de mapas 
no cenário onde a pose inicial dos agentes não é fornecida, como debatido na Seção \ref{sec:exp-unknown-initial-pose}. 
  \item \textbf{Recuperação após falsa associação}: Outro aspecto que merece ser contemplado num trabalho futuro é a recuperação após uma falsa associação de medida e \textit{landmark} e/ou a 
utilização de uma técnica de associação de dados mais robusta. Apesar da 
lista de \textit{landmarks} provisórias ter atenuado esse problema, não foi 
suficiente para resolvê-lo, como mostram as taxas de falha por divergência das 
simulações apresentadas nos experimentos das Seções \ref{sec:exp-single-robot}  e \ref{sec:exp-known-initial-pose}.
\end{itemize}




% o que aprendi