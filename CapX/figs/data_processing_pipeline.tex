\documentclass{standalone}

\usepackage{import}
\usepackage{standalone}
\usepackage{tikz}
\usetikzlibrary{positioning}

\usepackage{pgfplots}
\pgfplotsset{compat=1.12}

\begin{document}
    \begin{tikzpicture}[]
      \centering
      \node[] (sensorRawSignal) {\includestandalone[width=0.4\textwidth]{raw_sensor_data}};

      \node[below=of sensorRawSignal] (derivative) {\includestandalone[width=0.4\textwidth]{signal_derivative}};

      \node[left=of derivative, align=center] (derivativeText) {Cálculo da derivada do sinal\\e extração de pontos de mínimo\\e máximo fora dos valores limiares.};
      
      \path[-latex, ultra thick] (sensorRawSignal) edge (derivative);
      
      \node[align=center] (inputText) at (sensorRawSignal -| derivativeText) {\textbf{ENTRADA}\\Sequência de distâncias\\lidas pelo sensor.};

      \node[below=of derivative] (cartesianData) {\includestandalone[width=0.4\textwidth]{sensor_data_cartesian}};

      \path[-latex, ultra thick] (derivative) edge (cartesianData);

      \node[align=center] (cartesianDataText) at (cartesianData -| derivativeText) {Segmentação dos pontos entre\\os pontos de mínimo e máximo do\\sinal de derivada, transformação\\
      para coordenadas cartesianas.};

      \node[below=of cartesianData] (circleFit) {\includestandalone[width=0.4\textwidth]{circle_detection}};

      \path[-latex, ultra thick] (cartesianData) edge (circleFit);
      
      \node[align=center] (circleFitText) at (cartesianDataText |- circleFit) {\textbf{SAÍDA}\\Estimação dos círculos que\\descrevem os conjuntos de pontos,\\com $r > 6.5cm$ e $\sigma < \epsilon$};

      \node[circle, draw=black, ultra thick, left=of derivativeText] (2) {2};
      \node[circle, draw=black, ultra thick, left=of cartesianDataText] {3};
      \node[circle, draw=black, ultra thick, left=of circleFitText] {4};
      \node[circle, draw=black, ultra thick] at (2 |- inputText) {1};
    \end{tikzpicture}
\end{document}