\section{Matriz de informação aumentada a partir da matriz de covariância}
\label{app:dev-landmark-insertion-info-matrix}
Para derivar a inserção de \textit{landmark} no EIF/SEIF, vamos partir 
dos resultados da inserção de \textit{landmark} no EKF apresentados na 
Seção \ref{sec:ekf-slam-landmark-insertion}. Por comodidade, repetiremos 
a Equação \ref{eq:ekf-landmark-insertion-covariance} com o resultado 
do aumento da matriz de covariância:
\begingroup
\renewcommand{\arraystretch}{1.5}
\begin{equation*}
  \bsubvec{P}{t}^* = \eqCovarianceAugmented \tag{\ref{eq:ekf-landmark-insertion-covariance} repetida}
\end{equation*}
\endgroup
e aplicando a equivalência $\bvec{\Omega}^{-1} = \bvec{P}$, pode-se reescrever a expressão acima como:
% \newcommand{\eqOmegaInv}{\bsubvec{\Omega}{t}^{-1}}
\begingroup
\renewcommand{\arraystretch}{1.5}
\begin{equation}
  \brac{\bsubvec{\Omega}{t}^*}^{-1} = \begin{bmatrix}
    \eqOmegaInv & \eqOmegaInv \bsubvecT{F}{X} \\
    \bsubvec{F}{X} \eqOmegaInv & \bsubvec{F}{X}\eqOmegaInv\bsubvecT{F}{X} + \bsubvec{F}{Y} \bvec{Q} \bsubvecT{F}{Y}
  \end{bmatrix}
\end{equation}
Invertendo ambos os lados:
\begin{equation*}
  \bsubvec{\Omega}{t}^* = \begin{bmatrix}
    \bvec{X} & \bvec{Y} \\
    \bvec{Z} & \bvec{U}
  \end{bmatrix}
\end{equation*}
\endgroup

Pelo lema da inversão na forma de blocos (Anexo \ref{annex:blockwise-inversion-lemma}) temos:
\addtolength{\jot}{1.4pt}
\begingroup
\begin{align}
  &\begin{aligned}
    \bvec{U} &=  \brac{\bsubvec{F}{X}\eqOmegaInv\bsubvecT{F}{X} + \bsubvec{F}{Y} \bvec{Q} \bsubvecT{F}{Y} - \bsubvec{F}{X} \cancel{\eqOmegaInv \brac{\eqOmegaInv}^{-1}} \,\eqOmegaInv \bsubvecT{F}{X}}^{-1}\\
    &= \brac{\cancel{\bsubvec{F}{X}\eqOmegaInv\bsubvecT{F}{X}} + \bsubvec{F}{Y} \bvec{Q} \bsubvecT{F}{Y} - \cancel{\bsubvec{F}{X} \eqOmegaInv \bsubvecT{F}{X}}}^{-1}\\
    &= \eqU
  \end{aligned}\\
  &\begin{aligned}
    \bvec{Y} &= -\cancel{\brac{\eqOmegaInv}^{-1} \eqOmegaInv}\bsubvecT{F}{X} \bvec{U}\\
    &= \bsubvecT{F}{X} \parentheses{\eqU} \\
  \end{aligned}\\
  &\begin{aligned}
    \bvec{Z} &= -\bvec{U} \bsubvec{F}{X} \cancel{\eqOmegaInv \brac{\eqOmegaInv}^{-1}}\\
    &= -\parentheses{\eqU} \bsubvec{F}{X}
  \end{aligned}\\
  &\begin{aligned}
    \bvec{X} &= \brac{\eqOmegaInv}^{-1} + \cancel{\brac{\eqOmegaInv}^{-1} \eqOmegaInv}\bsubvecT{F}{X} \,\bvec{U} \, \bsubvec{F}{X} \cancel{\eqOmegaInv \brac{\eqOmegaInv}^{-1}}\\
    &= \bsubvec{\Omega}{t} + \bsubvecT{F}{X} \parentheses{\eqU} \bsubvec{F}{X}
  \end{aligned}
\end{align}
Por fim, temos que a matriz de informação aumentada é dada por:
\renewcommand{\arraystretch}{1.5}
\begin{equation}
  \bsubvec{\Omega}{t}^* = \begin{bmatrix}
    \bsubvec{\Omega}{t} + \bsubvecT{F}{X} \parentheses{\eqU} \bsubvec{F}{X} & -\bsubvecT{F}{X} \parentheses{\eqU} \\
    -\parentheses{\eqU} \bsubvec{F}{X} & \eqU
  \end{bmatrix} 
  \label{eq:seif-landmark-insertion-information-matrix-app}
\end{equation}
\endgroup