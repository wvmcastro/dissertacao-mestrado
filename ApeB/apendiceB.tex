\section{Lema da Inversão. Fórmula de  Sherman/Morrison}
\label{app:inversion-lemma}
A fórmula de Sherman/Morrison, também conhecida como lema da inversão especializado,
é definido a seguir, retirado de \cite[p.~50]{bongard2006probabilistic}.

\begin{lem}
Para qualquer matrizez quadradas invertíveis $\bvec{R}$ e $\bvec{Q}$ e qualquer matriz $\bvec{P}$ com dimensões apropriadas, o seguinte é verdadeiro:
\begin{equation}
\parentheses{\bvec{R} + \bvec{PQ}\bvecT{P}}^\mathbf{{-1}} = \bvec{R^{-1}} - \bvec{R^{-1}}\bvec{P}\parentheses{\bvec{Q^{-1}} + \bvecT{P}\bvec{R^{-1}P}}^\mathbf{{-1}}\bvecT{P}\bvec{R^{-1}}
\end{equation}
assumindo que todas as matrizes acima podem ser invertidas como definido na premissa.
\begin{proof}
Defina $\bvec{\Psi} = \parentheses{\bvec{Q^{-1}} + \bvecT{P}\bvec{R^{-1}}\bvec{P}}^\mathbf{-1}$. É suficiente mostrar que:
\begin{equation*}
\parentheses{\bvec{R^{-1}} - \bvec{R^{-1}}\bvec{P}\bvec{\Psi}\bvecT{P}\bvec{R^{-1}}} \parentheses{\bvec{R} + \bvec{PQ}\bvecT{P}} = \bvec{I}
\end{equation*}
Isso é mostrado através de uma série de manipulações:
\begin{equation}
\begin{aligned}
    &= \blue{\bvec{R^{-1} R}} + \bvec{R^{-1} P Q P^T} - \bvec{R^{-1} P \Psi P^T \magenta{R^{-1} R}} - \bvec{R^{-1} P \Psi P^T R^{-1} P Q P^T} \\
    &= \blue{\bvec{I}} + \bvec{R^{-1} P Q P^T} - \bvec{R^{-1} P \Psi P^T \magenta{I}} - \bvec{R^{-1} P \Psi P^T R^{-1} P Q P^T}\\
    &= \bvec{I} + \bvec{\magenta{R^{-1} P} Q P^T} - \bvec{\magenta{R^{-1} P} \Psi P^T} - \bvec{\magenta{R^{-1} P} \Psi P^T R^{-1} P Q P^T}\\
    &= \bvec{I} + \bvec{R^{-1} P} \left[\bvec{Q P^T} - \bvec{\Psi P^T} - \bvec{\Psi P^T R^{-1} P Q P^T}\right]\\
    &= \bvec{I} + \bvec{R^{-1} P} \left[\bvec{Q P^T} - \bvec{\Psi \blue{Q^{-1} Q} P^T} - \bvec{\Psi P^T R^{-1} P Q P^T}\right]\\
    &= \bvec{I} + \bvec{R^{-1} P} \left[\bvec{Q P^T} - \bvec{\magenta{\Psi} Q^{-1}\blue{ Q P^T}} - \bvec{\magenta{\Psi} P^T R^{-1} P \blue{Q P^T}}\right]\\
    &= \bvec{I} + \bvec{R^{-1} P} \left[\bvec{Q P^T} - \bvec{\magenta{\Psi}} \left(\bvec{ Q^{-1}} - \bvec{P^T R^{-1} P}\right)\bvec{\blue{Q P^T}}\right]\\
    &= \bvec{I} + \bvec{R^{-1} P} \left[\bvec{Q P^T} - \bvec{\Psi} \blue{\left(\bvec{ Q^{-1}} - \bvec{P^T R^{-1} P}\right)}\bvec{Q P^T}\right]\\
    &= \bvec{I} + \bvec{R^{-1} P} \left[\bvec{Q P^T} - \bvec{\Psi} \blue{\bvec{\Psi^{-1}}}\bvec{Q P^T}\right]\\
    &= \bvec{I} + \bvec{R^{-1} P} \left[\bvec{Q P^T} - \bvec{Q P^T}\right]\\
    &= \bvec{I} 
\end{aligned}
\end{equation}

\end{proof}
\label{lem:inversion-lemma}

\end{lem}