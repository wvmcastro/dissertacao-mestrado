O famoso Filtro de Kalman (KF) é uma técnica de estimação ótima para sistemas lineares com ruídos gaussianos, nele a distribuição de probabilidade do estado estimado é representada por uma gaussiana, parametrizada pelos momentos média e covariância. Ele foi desenvolvido simultaneamente em 1958 por Peter Swerling, e em 1960 por Rudolf Kalman \cite[p.~40]{bongard2006probabilistic}. Apesar de sua otimalidade ser garantida apenas para sistemas lineares, ele é aplicado em sistemas não lineares também. Para isso, é feita uma aproximação linear em torno da estimativa do estado atual do sistema, utilizando-se série de Taylor, e a premissa de que os termos de ordem maior ou igual a dois são desprezíveis.

Essa técnica derivada do KF para sistemas não lineares é conhecida como Filtro de Kalman Extendido (EKF). O EKF é muito utilizado em aplicações reais, pois a grande maioria dos sistemas reais são não lineares, como o movimento de um robô diferencial, por exemplo. Além disso o modelo de medida do sensor é, muitas vezes, uma função não linear do estado do sistema.

Neste capítulo, serão descritas as alterações necessárias no EKF clássico para que ele possa ser aplicado na resolução do problema de SLAM, estimando a pose do robô e a posição das \textit{landmarks}, o que é conhecido como EKF-SLAM. Além disso, também serão abordadas técnicas decorrentes do EKF-SLAM como EIF-SLAM e SEIF-SLAM, sendo este última a técnica de estimação utilizada neste trabalho.

\section{Filtro de Kalman Extendido}
Para que seja possível estimar o estado de um sistema utilizando-se KF, é 
necessário conhecer duas equações: a primeira, denominada modelo do sistema, 
modela a transição de estado do sistema a partir do estado anterior e da entrada aplicada, Eq. \ref{eq:system-model}; a segunda, chamada de modelo de medida, relaciona o estado do sistema com a medida esperada, gerada pelo sensor, Eq. \ref{eq:measurement-model}. 
\begin{align}
  \bsubvec{x}{t} &= \mb{g}(\bsubvec{x}{t-1}, \bsubvec{u}{t}) + \mb{\epsilon}_t
  \label{eq:system-model}\\
  \bsubvec{y}{t} &= \mb{h}(\bsubvec{x}{t}) + \mb{\delta}_t
  \label{eq:measurement-model}
\end{align}
Como o modelo do sistema $\mb{g}(\mathord{\bullet}, \mathord{\bullet})$, e o modelo de medida $\mb{h}(\mathord{\bullet}, \mathord{\bullet})$ não são exatos, suas incertezas e erros de modelagem são aproximados por ruídos gaussianos $\mb{\epsilon}_t \sim \normal{\bvec{0}}{\bsubvec{R}{t}}$ e $\mb{\delta}_t \sim \normal{\bvec{0}}{\bsubvec{Q}{t}}$. Quando $\mb{g}(\mathord{\bullet}, \mathord{\bullet})$ e/ou $\mb{h}(\mathord{\bullet}, \mathord{\bullet})$ não são lineares, o EKF pode ser utilizado para estimar o estado 
do sistema. 

As Equações de \ref{eq:ekf-prediction} até \ref{eq:ekf-estimate-error-covariance} definem o EKF \footnote{O leitor pode estanhar a Equação \ref{eq:ekf-estimate-error-covariance} do erro da estimativa. Normalmente ela é escrita na forma $\cov{t} = (\bvec{I} - \bsubvec{K}{t} \bsubvec{H}{t}) \predcov{t}$, porém de acordo com \cite[p.~ 73]{lewis2017optimal}, a forma em \ref{eq:ekf-estimate-error-covariance} é uma alternativa melhor na presença de erros de arredondamento, e é frequentemente utilizada em implementações de software.} para o sistema não linear acima. Onde $\bsubvec{G}{t}$ é o jacobiano do modelo do sistema no ponto $\mean{t-1}$, e $\bsubvec{H}{t}$ é o jacobiano do modelo de medida no ponto $\predmean{t}$.
\begin{align}
  \predmean{t} &= \mb{g}(\mean{t-1}, \bsubvec{u}{t})
  \label{eq:ekf-prediction}\\
  \predcov{t} &= \bsubvec{G}{t}\cov{t-1}\bsubvecT{G}{t} + \bsubvec{R}{t}
  \label{eq:ekf-prediction-error-covariance}\\
  \bsubvec{z}{t} &= \bsubvec{y}{t} - \mb{h}(\predmean{t})
  \label{eq:ekf-inovation} \\
  \bsubvec{Z}{t} &= \bsubvec{H}{t}\predcov{t}\bsubvecT{H}{t} + \bsubvec{Q}{t}
  \label{eq:ekf-inovation-covariance} \\
  \bsubvec{K}{t} &=   \predcov{t}\bsubvecT{H}{t} \bsubvec{Z}{t}^{-1}
  \label{eq:ekf-gain}\\
  \mean{t} &= \predmean{t} + \bsubvec{K}{t} \bsubvec{z}{t}
  \label{eq:ekf-update}\\
  \cov{t} &= \predcov{t} - \bsubvec{K}{t}\bsubvec{Z}{t}\bsubvecT{K}{t}
  \label{eq:ekf-estimate-error-covariance}
\end{align}
Porém, a resolução do problema de SLAM utilizando o EKF, não é uma aplicação 
direta das Equações acima. São necessárias algumas alterações, pois em SLAM o vetor de medidas tem tamanho variável. Esses detalhes e outras particularidades da aplicação do EKF em SLAM serão tratados na próxima Seção.

\section{EKF-SLAM}
Para aplicar o EKF na solução de SLAM, é necessário entender 
como o vetor de estados $\bvec{x}$ é composto (aqui o subíndice $t$ é omitido, pois não é importante para esta discussão). Como tanto a pose do 
robô, como o mapa são estimados, o vetor de estados é composto por ambos. 
\begin{equation}
  \bvec{x} = \begin{bmatrix}
    \bsubvec{x}{R} \\
    \bsubvec{x}{M}
  \end{bmatrix}
  \label{eq:slam-state-vector}
\end{equation}
O vetor $\bsubvec{x}{M}$, que representa o mapa, é composto pela posição $(x, y)$ das \textit{landmarks} identificadas. 
Seu tamanho é variável e cresce à medida que o robô navega pelo ambiente e 
mede novas \textit{landmarks}.
\begin{equation}
  \bsubvec{x}{M} = \begin{bmatrix}
    m_x^1\\
    m_y^1\\
    \vdots\\
    m_x^n\\
    m_y^n
  \end{bmatrix}
  \label{eq:feature-map}
\end{equation}

\subsection{EKF-SLAM: Predição (Movimento do robô)}
\label{sec:ekf-slam-prediction}
Em SLAM apenas uma parte do vetor de estados é variante no tempo, a pose do 
robô. Isso significa que apenas a porção $\bsubvec{x}{R}$ é alterada pela entrada $\bvec{u}$, logo o modelo do sistema consiste apenas do modelo de 
movimento do robô $\mb{g}_R$ concatenado com as posições das \textit{landmarks}:
\begin{equation}
  \bsubvec{x}{t} = \begin{bmatrix}
    \mb{g}_R(\bsubvec{x}{R, \,t}, \bsubvec{u}{t}) \\
    \bsubvec{x}{M}
  \end{bmatrix} + \begin{bmatrix} 
      \mb{\epsilon}_{R, \,t} \\
      \bvec{0}
  \end{bmatrix}
  \label{eq:slam-prediction}
\end{equation}
Portanto, o passo de predição do vetor média do EKF-SLAM torna-se:
\begin{equation}
  \predmean{t} = \begin{bmatrix}
    \mb{g}_R(\mean{R, \,t-1}, \bsubvec{u}{t}) \\
      \mean{M}
  \end{bmatrix}
\end{equation}
em termos de implementação, isso significa que apenas as posições de memória da 
pose são modificadas no vetor de estados. Dessa forma, o passo de predição do 
vetor de estados do EKF-SLAM 2D tem complexidade $\bigO{3}$ (constante), enquanto no EKF 
essa complexidade é $\bigO{n}$, onde $n$ é o tamanho do vetor de estados.

A matriz de covariância, $\cov{}$, também é parcialmente atualizada, pois o jacobiano do sistema na Equação \ref{eq:slam-prediction} possui forma esparsa:
\begin{equation}
  \bsubvec{G}{t} = \begin{bmatrix}
    \bsubvec{G}{R, \,t} & \bvec{0} \\
    \bvec{0} & \bvec{I}
  \end{bmatrix}
\end{equation}
Então, a Equação do erro de predição do EKF, em \ref{eq:ekf-prediction-error-covariance}, torna-se:
\renewcommand{\arraystretch}{1.5}
\begin{equation}
\begin{aligned}
  \predcov{t} &= \begin{bmatrix}
    \bsubvec{G}{R, \,t} & \bvec{0} \\
    \bvec{0} & \bvec{I}
  \end{bmatrix} \cov{t-1}  \begin{bmatrix}
    \bsubvec{G}{R, \,t} & \bvec{0} \\
    \bvec{0} & \bvec{I}  
  \end{bmatrix}^T + \begin{bmatrix}
      \bsubvec{R}{t}  & \bvec{0} \\ \bvec{0} & \bvec{0}
    \end{bmatrix} \\
  &= \begin{bmatrix}
    \bsubvec{G}{R, \,t} & \bvec{0} \\
    \bvec{0} & \bvec{I}
  \end{bmatrix} 
  \begin{bmatrix}
    \cov{RR, \, t-1} & \cov{RM, \, t-1} \\
    \cov{MR, \,t-1} & \cov{MM}
  \end{bmatrix}  
  \begin{bmatrix}
    \bsubvec{G}{R, \,t} & \bvec{0} \\
    \bvec{0} & \bvec{I}  
  \end{bmatrix}^T + \begin{bmatrix}
      \bsubvec{R}{t} & \bvec{0} \\ \bvec{0} & \bvec{0}
    \end{bmatrix} \\
  &= \begin{bmatrix}
    \bsubvec{G}{R,\,t} \cov{RR,\, t-1} \bsubvecT{G}{R, \,t} + \bsubvec{R}{t}&  \bsubvec{G}{R,\, t} \cov{RM,\, t-1}\\
    \cov{MR,\, t-1}\bsubvecT{G}{R,\, t} & \cov{MM} 
  \end{bmatrix} 
\end{aligned}
\label{eq:ekf-slam-prediction-error-covariance}
\end{equation}
\renewcommand{\arraystretch}{1.0}
a complexidade dessa operação é da ordem de $\bigO{n}$ por conta do termo $\bsubvec{G}{R,\, t} \cov{RM,\, t-1}$, enquanto no caso geral do EKF onde o jacobiano $\bsubvec{G}{t}$ é denso, essa complexidade é $\bigO{n^3}$, que é a complexidade prática da multiplicação de matrizes $n \times n$. A Figura 
\ref{fig:ekfslam-prediction} ilustra as porções do vetor de estados, e da 
matriz de covariância, modificadas no passo de predição do EKF-SLAM.

\begin{figure}[h]
  \centering
  \includestandalone[width=.45\textwidth]{figs/ekfslam-prediction}
  \caption{Partes modificadas do vetor média e da matriz de covariância durante o movimento do robô. O vetor média é representado pela barra na esquerda, e a matriz de covariância pelo quadrado na direita. As partes modificadas, em tons de cinza, correspondem ao estado do robô $\mean{R}$  e sua autocovariância $\cov{RR}$ (cinza escuro), e às covariâncias cruzadas, $\cov{RM}$ e $\cov{MR}$, entre o robô e o mapa (cinza claro). Note que as partes correspondentes ao mapa, $\mean{M}$ e $\cov{MM}$, 
  permanecem inalteradas (branco). Adaptado de \cite[p.~10]{jsola}.}
  \label{fig:ekfslam-prediction}
\end{figure}


\subsection{EKF-SLAM: Atualização}
\label{sec:ekf-slam-update}
Assim como na predição, no passo de atualização há algumas particularidades 
que devem ser levadas em conta no EKF-SLAM. Ao contrário de um sistema convencional, em SLAM o vetor de medidas é variável, seu tamanho depende da 
quantidade de \textit{landmarks} que vão sendo avistadas pelo robô enquanto 
ele navega pelo ambiente. Ou seja, no EKF-SLAM o vetor de medidas é sempre 
``incompleto'', e normalmente a inovação $\bsubvec{z}{t}$ é calculada para cada 
medida de maneira individual, e é denotada por $\bsubvec{z}{t}^j$.
\begin{equation}
  \bsubvec{z}{t}^j = \bsubvec{y}{t}^j - \mb{h}^j(\predmean{t})
  \label{eq:ekf-slam-innovation}
\end{equation}
Além disso, como o jacobiano do modelo de medida na Equação \ref{eq:measurement-model-jacobian-full} é esparso, o cálculo da covariância da inovação pode ser obtido por:
\renewcommand{\arraystretch}{1.5}
\begin{equation}
  \bsubvec{Z}{t}^j = \begin{bmatrix}
    \bsubvec{H}{R}^j & \bsubvec{H}{M}^j
  \end{bmatrix}
  \begin{bmatrix}
    \predcov{RR} & \predcov{RM_j} \\
    \predcov{RM_j}^T & \predcov{M_j M_j}
  \end{bmatrix}
  \begin{bmatrix}
    \bsubvec{H}{R}^j \\ \bsubvec{H}{M}^j
  \end{bmatrix} + \bsubvec{Q}{t}
  \label{eq:ekf-slam-innovation-covariance}
\end{equation}
\renewcommand{\arraystretch}{1.0}
As dimensões da inovação e das matrizes na Equação acima são constantes e dependem apenas da dimensão da pose do robô, e da dimensão da medida. 
Portanto, aqui as complexidades dos cálculos da inovação $\bsubvec{z}{t}^j$, e de sua covariância $\bsubvec{Z}{t}^j$ são constantes, enquanto no EKF essas complexidades são $\bigO{m}$ e $\bigO{nm^2}$, respectivamente, onde $m$ é o tamanho do vetor de medidas. Embora, claro, aqui esse cálculo de complexidade constante deva ser realizado para cada observação presente no vetor de medidas, ou seja, no EKF-SLAM o cálculo da
inovação, e de sua covariância possuem complexidade linear no número de medidas obtidas.

O cálculo do Ganho de Kalman, $\bsubvec{K}{t}$, também é influenciado pelo 
tamanho constante da matriz de covariância da inovação ($2\times 2$, no caso deste trabalho), Equação \ref{eq:ekf-slam-innovation-covariance}, e pela esparsidade do jacobiano do modelo de medida, na Equação \ref{eq:measurement-model-jacobian-full}. Ademais, se todos os cálculos triviais de multiplicação por zero não forem feitos, a complexidade do cálculo do Ganho de Kalman, $\bsubvec{K}{t}^j$, é $\bigO{n}$ no EKF-SLAM.

Por fim, as complexidades da atualização e sua matriz de covariância, Equações \ref{eq:ekf-update} e \ref{eq:ekf-estimate-error-covariance}, são $\bigO{n}$ e $\bigO{n^2}$, respectivamente. A Figura \ref{fig:ekf-slam-innovation} mostra as porções da 
do vetor de estados e da matriz de covariância utilizadas no cálculo da inovação e de sua matriz de covariância.

\begin{figure}[h]
  \centering
  \includestandalone[width=.45\textwidth]{figs/ekfslam-innovation}
  \caption{Partes utilizadas do vetor média e da matriz de covariância durante o cálculo da inovação, quando uma \textit{landmark} é observada. O vetor média é representado pela barra na esquerda, e a matriz de covariância pelo quadrado na direita. As porções utilizadas, em tons de cinza, correspondem ao estado do robô $\mean{R}$ e à posição da \textit{landmark} $\bvec{m}^j$, e suas autocovariâncias $\cov{RR}$ e $\cov{M^j M^j}$ (cinza escuro), e às covariâncias cruzadas, $\cov{RM^j}$ e $\cov{M^jR}$, entre o robô e a j-ésima \textit{landmark} (cinza claro). Adaptado de \cite[p.~8]{jsola}.}
  \label{fig:ekf-slam-innovation}
\end{figure}

A Figura \ref{fig:ekf-slam-update} deixa claro que todos os elementos do vetor média e da 
matriz de covariâncias são atualizados pelas Equações \ref{eq:ekf-update} e \ref{eq:ekf-estimate-error-covariance}, mesmo o cálculo da inovação sendo esparso. Isso ocorre porque no EKF todas as \textit{landmarks} são 
correlacionadas, mesmo que muitas dessas correlações sejam próximas de zero. 
Esse tipo de correlação ``fraca'' será explorada pelo Filtro de Informação Extendido Esparso, a fim de obter-se um algoritmo de estimação mais eficiente.

\begin{figure}[h]
  \centering
  \includestandalone[width=.45\textwidth]{figs/ekfslam-update}
  \caption{O vetor média e a matriz de covariâncias são completamente atualizados durante a observação de uma \textit{landmark}. Retirado de \cite[p.~8]{jsola}.}
  \label{fig:ekf-slam-update}
\end{figure}

\subsection{EKF-SLAM: Inserção de \textit{landmark} (aumento do vetor de estados)}
Nas Seções anteriores, \ref{sec:ekf-slam-prediction} e \ref{sec:ekf-slam-update}, foram tratadas as diferenças 
do EKF-SLAM para o EKF, nas já conhecidas pelo usuário comum do EKF, etapas de predição e atualização. No entanto, em EKF-SLAM uma nova operação aparece: A etapa de inserção de \textit{landmark}. Ela ocorre quando o robô observa uma \textit{landmark} que ainda não está no mapa, $\bsubvec{x}{M}$, e portanto não é possível calcular a inovação na Equação \ref{eq:ekf-slam-innovation}. Nesse caso, a nova \textit{landmark} deve ser adicionada ao vetor média e à matriz de covariância, aumentando a dimensão do sistema.

Para adicionar uma nova \textit{landmark} no vetor de estado, será definida 
a função $\mb{\sigma}(\bullet, \bullet)$, ela gera um novo vetor de estados 
que é resultado da concatenação do vetor atual, com a posição da nova 
\textit{landmark} calculada pelo modelo de medida inverso, descrito na Seção \ref{sec:inverse-measurement-model}, a partir da leitura $\bvec{y}^j$.
\begin{equation}
      \mb{\sigma}(\bsubvec{x}{t}, \bvec{y}^j) = \begin{bmatrix}
        \bsubvec{x}{t} \\
        \mb{g}(\bsubvec{x}{t}, \bvec{y}{^j})
      \end{bmatrix}
\end{equation}
Porém, não basta apenas adicionar a nova \textit{landmark} no vetor de estados, é necessário adicioná-la também na matriz de covariâncias. Quando 
o robô observa uma nova \textit{landmark}, é esperado que o erro de estimação 
da posição dessa nova \textit{landmark} seja influenciado pelo erro da 
pose do robô, no momento da leitura, e pelo erro de de medição do sensor. 

Inicializar a covariância da nova \textit{landmark} com $\infty$ (ou 
números muito grandes), como indicado em 
\cite[p.~317]{bongard2006probabilistic}, pode ser injusto. Portanto devemos 
calcular o erro, $\mb{\alpha}$, da nova estimativa do vetor aumentado, de 
maneira análoga à forma como é feita nos passos de predição e atualização 
do EKF. 
\begin{equation}
\begin{aligned}
   \mb{\alpha} &= \bsubvec{x}{t}^* - \mean{t}^*\\
   &= \begin{bmatrix}
       \bsubvec{x}{t} \\ \mb{f}(\bsubvec{x}{t}, \bvec{y})
   \end{bmatrix} - \begin{bmatrix}
       \mean{t} \\ \mb{f}(\mean{t}, \bvec{y}^j)
   \end{bmatrix} = \begin{bmatrix}
       \bsubvec{x}{t} - \mean{t}\\
       \mb{f}(\bsubvec{x}{t}, \bvec{y}) - \mb{f}(\mean{t}, \bvec{y}^j)
   \end{bmatrix}\\
   &= \begin{bmatrix}
       \mb{\eta}_t \\
       \cancel{\mb{f}(\mean{t}, \bvec{y}^j)} + 
       \bvec{F}_X \mb{\eta}_t + \bvec{F}_Y \mb{\delta}
       - \cancel{\mb{f}(\mean{t}, \bvec{y}^j)}
   \end{bmatrix} \\
    &= \begin{bmatrix}
       \mb{\eta}_t \\
       \bvec{F}_X \mb{\eta}_t + \bvec{F}_Y \mb{\delta} 
   \end{bmatrix} \small\text{ Onde $\mb{\eta}_t = \bsubvec{x}{t} - \mean{t}$ e $\mb{\delta} = \bvec{y} - \bvec{y}^j$}
\end{aligned}
\end{equation}
A matriz de covariância do sistema aumentado, $\cov{t}^*$, é obtida por
\renewcommand{\arraystretch}{1.5}
\begin{equation}
\begin{aligned}
  \cov{t}^* &= \expv{\mb{\alpha} \,\mb{\alpha}^T} \\
  &= \begin{bmatrix}
       \cov{t} & \cov{t} \bvec{F}_X^T\\
       \cov{t} \bvec{F}_X &  \bvec{F}_X \cov{t} \bvec{F}_X^T 
       + \bvec{F}_Y \bvec{Q} \bvec{F}_Y^T
    \end{bmatrix}
\end{aligned}
\end{equation}
\renewcommand{\arraystretch}{1}
portanto, a matriz de covariância do sistema aumentado é a matriz de 
covariância do sistema antes da inserção da nova \textit{landmark}, 
concatenada as covariâncias cruzadas $\cov{t} \bvec{F}_X^T$ e $\bvec{F}_X \cov{t}$, e com a covariância $\bvec{F}_X \cov{t} \bvec{F}_X^T + \bvec{F}_Y 
\bvec{Q} \bvec{F}_Y^T$, da nova \textit{landmark} inserida no mapa. 

Vale notar que o jacobiano $\bvec{F}_X$ descrito na Equação \ref{eq:inverse-measurement-model-jacobian-state-part} é esparso, logo a complexidade de $\cov{t}\bvec{F}_x$ pode ser reduzida de $\bigO{n^2}$ para $\bigO{n}$ se todos os cálculos inúteis forem ignorados, logo toda a operação de inserção de \textit{landmark} tem custo $\bigO{n}$. A Figura \ref{fig:ekf-slam-landmark-insertion} mostra o vetor média e a matriz de covariância com as novas inserções destacadas.

\begin{figure}[h]
  \centering
  \includestandalone[width=.6\textwidth]{figs/ekfslam-landmark-insertion}
  \caption{Vetor média e matriz de covariância aumentados após inserção de nova \textit{landmark}. As partes adicionadas, em cinza, correspondem às covariâncias cruzadas entre a nova landmark e o vetor de estados anterior (cinza claro), e à média da nova \textit{landmark} e sua covariância (cinza escuro). Adaptado de \cite[p.~11]{jsola}.}
  \label{fig:ekf-slam-landmark-insertion}
\end{figure}

\subsection{EKF-SLAM: Algoritmo}

\begin{algorithm}
  \setstretch{1.5}
  \caption{Etapa de predição do EKF-SLAM}
  \label{alg:ekf-slam-prediction}
\begin{algorithmic}[1]
\Procedure{EKF-SLAM-Predição}{$\mean{t-1}, \cov{t-1}, \bsubvec{u}{t}$}
  \State $\predmean{t} \gets \begin{bmatrix}
    \mb{g}_R(\mean{R,t-1}, \bsubvec{u}{t}) \\ \mean{M}
  \end{bmatrix}$
  \State $\predcov{RR,t} \gets \bsubvec{G}{R, t}\cov{RR, t-1} \bsubvecT{G}{R,t}$
  \State $\predcov{RM, t} \gets \bsubvec{G}{R, t} \cov{RM, t-1}$
  \State $\predcov{t} \gets \begin{bmatrix}
    \predcov{RR, t} & \predcov{RM, t} \\
    \predcov{RM, t}^T & \cov{MM}
  \end{bmatrix}$
  \State \Return $\predmean{t}, \predcov{t}$
\EndProcedure
\end{algorithmic}
\end{algorithm}

\begin{algorithm}[h]
  \setstretch{1.5}
  \caption{Etapa de atualização do EKF-SLAM}
  \label{alg:ekf-slam-update}
\begin{algorithmic}[1]
  \Procedure{EKF-SLAM-Atualizaçã}{$\predmean{t}, \predcov{t}, \bvec{y}, j = \text{índice da \textit{landmark}}$}
  \State $\bvec{z} \gets \bvec{y} - \mb{h}^j(\predmean{t})$
  \State $\bvec{Z} \gets \begin{bmatrix}
    \bsubvec{H}{R}^j & \bsubvec{H}{M}^j
  \end{bmatrix}
  \begin{bmatrix}
    \predcov{RR} & \predcov{RM_j} \\
    \predcov{RM_j}^T & \predcov{M_j M_j}
  \end{bmatrix}
  \begin{bmatrix}
    \bsubvec{H}{R}^j \\ \bsubvec{H}{M}^j
  \end{bmatrix} + \bsubvec{Q}{t}$
  \State $\bvec{K} \gets \begin{bmatrix} \cov{RR, t} & \cov{RM^j, t} \\ 
    \cov{MR, t} & \cov{MM^j, t} 
  \end{bmatrix}
  \begin{bmatrix}
    \brac{\bsubvec{H}{R}^j}^T \\ \brac{\bsubvec{H}{M}^j}^T
  \end{bmatrix} \bvec{Z}^{-1}$
  \State $\mean{t} \gets \bvec{K} \bvec{z}$
  \State $\cov{t} \gets \predcov{t} - \bvec{K}\bvec{Z}\bvecT{K}$
  \State \Return $\mean{t}, \cov{t}$
  \EndProcedure
\end{algorithmic}
\end{algorithm}

\section{Filtro de Informação Extendido (EIF)}

\section{SEIF-SLAM}

\section{Conclusão do capítulo}