Neste capítulo serão descritas as alterações necessárias no clássico Filtro de Kalman Extendido (EKF) para que ele possa ser aplicado na resolução do problema de SLAM, resultando na técnica EKF-SLAM. Além disso, também serão abordadas técnicas decorrentes da EKF-SLAM como EIF-SLAM e SEIF-SLAM, sendo esta última a técnica de estimação utilizada neste trabalho.

\section{Filtro de Kalman Extendido}
O Filtro de Kalman (KF)
As equações do EKF clássico, para sistemas discretos, são as equações de \ref{eq:ekf-prediction} até \ref{eq:ekf-estimate-error-covariance}.
\begin{align}
  \predmean{t} &= \mb{g}(\mean{t-1}, \bsubvec{u}{t})
  \label{eq:ekf-prediction}\\
  \predcov{t} &= \bsubvec{G}{t}\cov{t-1}\bsubvecT{G}{t} + \bsubvec{R}{t}
  \label{eq:ekf-prediction-error-covariance}\\
  \bsubvec{K}{t} &=   \predcov{t}\bsubvecT{H}{t} (\bsubvec{H}{t}\predcov{t}\bsubvecT{H}{t} + \bsubvec{Q}{t})^{-1}
  \label{eq:ekk-gain}\\
  \mean{t} &= \predmean{t} + \bsubvec{K}{t}(\bsubvec{y}{t} - \mb{h}(\predmean{t}))
  \label{eq:ekf-update}\\
  \cov{t} &= (\bvec{I} - \bsubvec{K}{t}\bsubvec{H}{t})\,\predcov{t}\,(\bvec{I} - \bsubvec{K}{t}\bsubvec{H}{t})^T + \bsubvec{K}{t}\bsubvec{Q}{t}\bsubvecT{K}{t}
  \label{eq:ekf-estimate-error-covariance}
\end{align}

\section{EKF-SLAM}
pipi popopopo