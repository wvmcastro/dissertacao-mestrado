O famoso Filtro de Kalman (KF) é uma técnica de estimação ótima para sistemas lineares com ruídos gaussianos, nele a distribuição de probabilidade do estado estimado é representada por uma gaussiana, 
Eq. \ref{eq:moments-gaussian-probabilistic-distribution}, parametrizada pelos momentos média e covariância. Ele foi desenvolvido simultaneamente em 1958 por Peter Swerling, e em 1960 por Rudolf Kalman \cite[p.~40]{bongard2006probabilistic}. Apesar de sua otimalidade ser garantida apenas para sistemas lineares, ele é aplicado em sistemas não lineares também. Para isso, é feita uma aproximação linear em torno da estimativa do estado atual do sistema, utilizando-se série de Taylor, e a premissa de que os termos de ordem maior ou igual a dois são desprezíveis.

\begin{equation}
  \pr{\bvec{x}} = \boldsymbol{\eta} \exp\parentheses{-\frac{1}{2} 
  \parentheses{\bvec{x} - \mean{}}^T \cov{}^{-1}\parentheses{\bvec{x} - \mean{}}}
  \label{eq:moments-gaussian-probabilistic-distribution}
\end{equation}

Essa técnica derivada do KF para sistemas não lineares é conhecida como Filtro de Kalman Extendido (EKF). O EKF é muito utilizado em aplicações reais, pois a grande maioria dos sistemas reais são não lineares, como o movimento de um robô diferencial, por exemplo. Além disso o modelo de medida do sensor é, muitas vezes, uma função não linear do estado do sistema.

Neste capítulo, serão descritas as alterações necessárias no EKF clássico para que ele possa ser aplicado na resolução do problema de SLAM, estimando a pose do robô e a posição das \textit{landmarks}, o que é conhecido como EKF-SLAM. Além disso, também serão abordadas técnicas decorrentes do EKF-SLAM como EIF-SLAM e SEIF-SLAM, sendo esta última a técnica de estimação utilizada neste trabalho.

\section{Filtro de Kalman Extendido}
\label{sec:ekf}
Para que seja possível estimar o estado de um sistema utilizando-se KF, é 
necessário conhecer duas equações: a primeira, denominada modelo do sistema, 
modela a transição de estado do sistema a partir do estado anterior e da entrada aplicada, Eq. \ref{eq:system-model}; a segunda, chamada de modelo de medida, relaciona o estado do sistema com a medida esperada, gerada pelo sensor, Eq. \ref{eq:measurement-model}. 
\begin{align}
  \bsubvec{x}{t} &= \mb{g}(\bsubvec{x}{t-1}, \bsubvec{u}{t}) + \mb{\epsilon}_t
  \label{eq:system-model}\\
  \bsubvec{y}{t} &= \mb{h}(\bsubvec{x}{t}) + \mb{\delta}_t
  \label{eq:measurement-model}
\end{align}
Como o modelo do sistema $\mb{g}(\mathord{\bullet}, \mathord{\bullet})$, e o modelo de medida $\mb{h}(\mathord{\bullet}, \mathord{\bullet})$ não são exatos, suas incertezas e erros de modelagem são aproximados por ruídos gaussianos $\mb{\epsilon}_t \sim \normal{\bvec{0}}{\bsubvec{R}{t}}$ e $\mb{\delta}_t \sim \normal{\bvec{0}}{\bsubvec{Q}{t}}$. Quando $\mb{g}(\mathord{\bullet}, \mathord{\bullet})$ e/ou $\mb{h}(\mathord{\bullet}, \mathord{\bullet})$ não são lineares, o EKF pode ser utilizado para estimar o estado 
do sistema. 

As Equações de \ref{eq:ekf-prediction} até \ref{eq:ekf-estimate-error-covariance} definem o EKF \footnote{O leitor pode estanhar a Equação \ref{eq:ekf-estimate-error-covariance} do erro da estimativa. Normalmente ela é escrita na forma $\cov{t} = (\bvec{I} - \bsubvec{K}{t} \bsubvec{H}{t}) \predcov{t}$, porém de acordo com \cite[p.~ 73]{lewis2017optimal}, a forma em \ref{eq:ekf-estimate-error-covariance} é uma alternativa melhor na presença de erros de arredondamento, e é frequentemente utilizada em implementações de software.} para o sistema não linear acima. Onde $\bsubvec{G}{t}$ é o jacobiano do modelo do sistema no ponto $\mean{t-1}$, e $\bsubvec{H}{t}$ é o jacobiano do modelo de medida no ponto $\predmean{t}$.
\begin{align}
  \predmean{t} &= \mb{g}(\mean{t-1}, \bsubvec{u}{t})
  \label{eq:ekf-prediction}\\
  \predcov{t} &= \bsubvec{G}{t}\cov{t-1}\bsubvecT{G}{t} + \bsubvec{R}{t}
  \label{eq:ekf-prediction-error-covariance}\\
  \bsubvec{z}{t} &= \bsubvec{y}{t} - \mb{h}(\predmean{t})
  \label{eq:ekf-inovation} \\
  \bsubvec{Z}{t} &= \bsubvec{H}{t}\predcov{t}\bsubvecT{H}{t} + \bsubvec{Q}{t}
  \label{eq:ekf-inovation-covariance} \\
  \bsubvec{K}{t} &=   \predcov{t}\bsubvecT{H}{t} \bsubvec{Z}{t}^{-1}
  \label{eq:ekf-gain}\\
  \mean{t} &= \predmean{t} + \bsubvec{K}{t} \bsubvec{z}{t}
  \label{eq:ekf-update}\\
  \cov{t} &= \predcov{t} - \bsubvec{K}{t}\bsubvec{Z}{t}\bsubvecT{K}{t}
  \label{eq:ekf-estimate-error-covariance}
\end{align}
Porém, a resolução do problema de SLAM utilizando o EKF, não é uma aplicação 
direta das Equações acima. São necessárias algumas alterações, pois em SLAM o vetor de medidas tem tamanho variável. Esses detalhes e outras particularidades da aplicação do EKF em SLAM serão tratados na próxima Seção.

\section{EKF-SLAM}
Para aplicar o EKF na solução de SLAM, é necessário entender 
como o vetor de estados $\bvec{x}$ é composto (aqui o subíndice $t$ é omitido, pois não é importante para esta discussão). Como tanto a pose do 
robô, como o mapa são estimados, o vetor de estados é composto por ambos. 
\begin{equation}
  \bvec{x} = \begin{bmatrix}
    \bsubvec{x}{R} \\
    \bsubvec{x}{M}
  \end{bmatrix}
  \label{eq:slam-state-vector}
\end{equation}
O vetor $\bsubvec{x}{M}$, que representa o mapa, é composto pela posição $(x, y)$ das \textit{landmarks} identificadas. 
Seu tamanho é variável e cresce à medida que o robô navega pelo ambiente e 
mede novas \textit{landmarks}.
\begin{equation}
  \bsubvec{x}{M} = \begin{bmatrix}
    m_x^1\\
    m_y^1\\
    \vdots\\
    m_x^n\\
    m_y^n
  \end{bmatrix}
  \label{eq:feature-map}
\end{equation}

Utilizando a definição do vetor de estados do EKF-SLAM acima, as próximas três Seções (\ref{sec:ekf-slam-prediction}, \ref{sec:ekf-slam-update} e \ref{sec:ekf-slam-landmark-insertion}) descrevem as alterações necessários e/ou desejáveis no EKF para sua aplicação em SLAM. O algoritmo completo pode ser encontrado no Apêndice \ref{app:alg-ekf-slam}.

\subsection{EKF-SLAM: Predição (Movimento do robô)}
\label{sec:ekf-slam-prediction}
Em SLAM apenas uma parte do vetor de estados é variante no tempo, a pose do 
robô. Isso significa que apenas a porção $\bsubvec{x}{R}$ é alterada pela entrada $\bvec{u}$, logo o modelo do sistema consiste apenas no modelo de 
movimento do robô $\mb{g}_R$ concatenado com as posições das \textit{landmarks}:
\begin{equation}
  \bsubvec{x}{t} = \begin{bmatrix}
    \mb{g}_R(\bsubvec{x}{R, \,t}, \bsubvec{u}{t}) \\
    \bsubvec{x}{M}
  \end{bmatrix} + \begin{bmatrix} 
      \mb{\epsilon}_{R, \,t} \\
      \bvec{0}
  \end{bmatrix}
  \label{eq:slam-prediction}
\end{equation}
Portanto, o passo de predição do vetor média do EKF-SLAM torna-se:
\begin{equation}
  \predmean{t} = \begin{bmatrix}
    \mb{g}_R(\mean{R, \,t-1}, \bsubvec{u}{t}) \\
      \mean{M}
  \end{bmatrix}
\end{equation}
Em termos de implementação, isso significa que apenas as posições de memória da 
pose são modificadas no vetor de estados. Dessa forma, o passo de predição do 
vetor de estados do EKF-SLAM 2D tem complexidade $\bigO{3}$ (constante), enquanto no EKF 
essa complexidade é $\bigO{n}$, onde $n$ é o tamanho do vetor de estados.

A matriz de covariância, $\cov{}$, também é parcialmente atualizada, pois o jacobiano do sistema na Equação \ref{eq:slam-prediction} possui forma esparsa:
\newcommand{\slamsystemjacobian}{
  \bsubvec{G}{t} = \begin{bmatrix}
    \bsubvec{G}{R, \,t} & \bvec{0} \\
    \bvec{0} & \bvec{I}
  \end{bmatrix}
}
\begin{equation}
  \slamsystemjacobian
  \label{eq:slam-system-jacobian}
\end{equation}
Então, a Equação do erro de predição do EKF, em \ref{eq:ekf-prediction-error-covariance}, torna-se:
\renewcommand{\arraystretch}{1.5}
\begin{equation}
\begin{aligned}
  \predcov{t} &= \begin{bmatrix}
    \bsubvec{G}{R, \,t} & \bvec{0} \\
    \bvec{0} & \bvec{I}
  \end{bmatrix} \cov{t-1}  \begin{bmatrix}
    \bsubvec{G}{R, \,t} & \bvec{0} \\
    \bvec{0} & \bvec{I}  
  \end{bmatrix}^T + \begin{bmatrix}
      \bsubvec{R}{R, t}  & \bvec{0} \\ \bvec{0} & \bvec{0}
    \end{bmatrix} \\
  &= \begin{bmatrix}
    \bsubvec{G}{R, \,t} & \bvec{0} \\
    \bvec{0} & \bvec{I}
  \end{bmatrix} 
  \begin{bmatrix}
    \cov{RR, \, t-1} & \cov{RM, \, t-1} \\
    \cov{MR, \,t-1} & \cov{MM}
  \end{bmatrix}  
  \begin{bmatrix}
    \bsubvec{G}{R, \,t} & \bvec{0} \\
    \bvec{0} & \bvec{I}  
  \end{bmatrix}^T + \begin{bmatrix}
      \bsubvec{R}{R, t} & \bvec{0} \\ \bvec{0} & \bvec{0}
    \end{bmatrix} \\
  &= \begin{bmatrix}
    \bsubvec{G}{R,\,t} \cov{RR,\, t-1} \bsubvecT{G}{R, \,t} + \bsubvec{R}{R, t}&  \bsubvec{G}{R,\, t} \cov{RM,\, t-1}\\
    \cov{MR,\, t-1}\bsubvecT{G}{R,\, t} & \cov{MM} 
  \end{bmatrix} 
\end{aligned}
\label{eq:ekf-slam-prediction-error-covariance}
\end{equation}
\renewcommand{\arraystretch}{1.0}
A complexidade dessa operação é da ordem de $\bigO{n}$ por conta do termo $\bsubvec{G}{R,\, t} \cov{RM,\, t-1}$, enquanto no caso geral do EKF onde o jacobiano $\bsubvec{G}{t}$ é denso, essa complexidade é $\bigO{n^3}$, que é a complexidade prática da multiplicação de matrizes $n \times n$. A Figura 
\ref{fig:ekfslam-prediction} ilustra as porções do vetor de estados, e da 
matriz de covariância, modificadas no passo de predição do EKF-SLAM.

\begin{figure}[h]
  \centering
  \includestandalone[width=.45\textwidth]{figs/ekfslam-prediction}
  \caption{Partes modificadas do vetor média e da matriz de covariância durante o movimento do robô. O vetor média é representado pela barra na esquerda, e a matriz de covariância pelo quadrado na direita. As partes modificadas, em tons de cinza, correspondem ao estado do robô $\mean{R}$  e sua autocovariância $\cov{RR}$ (cinza escuro), e às covariâncias cruzadas, $\cov{RM}$ e $\cov{MR}$, entre o robô e o mapa (cinza claro). Note que as partes correspondentes ao mapa, $\mean{M}$ e $\cov{MM}$, 
  permanecem inalteradas (branco). Adaptado de \cite[p.~10]{jsola}.}
  \label{fig:ekfslam-prediction}
\end{figure}


\subsection{EKF-SLAM: Atualização}
\label{sec:ekf-slam-update}
Assim como na predição, no passo de atualização há algumas particularidades 
que devem ser levadas em conta no EKF-SLAM. Ao contrário de um sistema convencional, em SLAM o vetor de medidas é variável, seu tamanho depende da 
quantidade de \textit{landmarks} que vão sendo avistadas pelo robô enquanto 
ele navega pelo ambiente. Ou seja, no EKF-SLAM o vetor de medidas é sempre 
``incompleto'', e normalmente a inovação $\bsubvec{z}{t}$ é calculada para cada 
medida de maneira individual, e é denotada por $\bsubvec{z}{t}^j$.
\begin{equation}
  \bsubvec{z}{t}^j = \bsubvec{y}{t}^j - \mb{h}^j(\predmean{t})
  \label{eq:ekf-slam-innovation}
\end{equation}
Além disso, como o jacobiano do modelo de medida na Equação \ref{eq:measurement-model-jacobian-full} é esparso, o cálculo da covariância da inovação pode ser obtido por:
\renewcommand{\arraystretch}{1.5}
\begin{equation}
  \bsubvec{Z}{t}^j = \begin{bmatrix}
    \bsubvec{H}{R}^j & \bsubvec{H}{M}^j
  \end{bmatrix}
  \begin{bmatrix}
    \predcov{RR} & \predcov{RM_j} \\
    \predcov{RM_j}^T & \predcov{M_j M_j}
  \end{bmatrix}
  \begin{bmatrix}
    \bsubvec{H}{R}^j \\ \bsubvec{H}{M}^j
  \end{bmatrix} + \bsubvec{Q}{t}
  \label{eq:ekf-slam-innovation-covariance}
\end{equation}
\renewcommand{\arraystretch}{1.0}
As dimensões da inovação e das matrizes na Equação acima são constantes e dependem apenas da dimensão da pose do robô, e da dimensão da medida. 
Portanto, aqui as complexidades dos cálculos da inovação $\bsubvec{z}{t}^j$, e de sua covariância $\bsubvec{Z}{t}^j$ são constantes, enquanto no EKF essas complexidades são $\bigO{m}$ e $\bigO{nm^2}$, respectivamente, onde $m$ é o tamanho do vetor de medidas. Embora, aqui esse cálculo de complexidade constante deve ser repetido para cada observação presente no vetor de medidas, ou seja, no EKF-SLAM o cálculo da
inovação, e de sua covariância possuem complexidade linear no número de medidas obtidas.

O cálculo do Ganho de Kalman, $\bsubvec{K}{t}$, também é influenciado pelo 
tamanho constante da matriz de covariância da inovação ($2\times 2$, no caso deste trabalho), Equação \ref{eq:ekf-slam-innovation-covariance}, e pela esparsidade do jacobiano do modelo de medida, na Equação \ref{eq:measurement-model-jacobian-full}. Ademais, se todos os cálculos triviais de multiplicação por zero não forem feitos, a complexidade do cálculo do Ganho de Kalman, $\bsubvec{K}{t}^j$, é $\bigO{n}$ no EKF-SLAM.

Por fim, as complexidades da atualização e sua matriz de covariância, Equações \ref{eq:ekf-update} e \ref{eq:ekf-estimate-error-covariance}, são $\bigO{n}$ e $\bigO{n^2}$, respectivamente. A Figura \ref{fig:ekf-slam-innovation} mostra as porções 
do vetor de estados e da matriz de covariância do sistema SLAM, utilizadas no cálculo da inovação e de sua matriz de covariância.

\begin{figure}[h]
  \centering
  \includestandalone[width=.45\textwidth]{figs/ekfslam-innovation}
  \caption{Partes utilizadas do vetor média e da matriz de covariância durante o cálculo da inovação, quando uma \textit{landmark} é observada. O vetor média é representado pela barra na esquerda, e a matriz de covariância pelo quadrado na direita. As porções utilizadas, em tons de cinza, correspondem ao estado do robô $\mean{R}$ e à posição da \textit{landmark} $\bvec{m}^j$, e suas autocovariâncias $\cov{RR}$ e $\cov{M^j M^j}$ (cinza escuro), e às covariâncias cruzadas, $\cov{RM^j}$ e $\cov{M^jR}$, entre o robô e a j-ésima \textit{landmark} (cinza claro). Adaptado de \cite[p.~8]{jsola}.}
  \label{fig:ekf-slam-innovation}
\end{figure}

A Figura \ref{fig:ekf-slam-update} deixa claro que todos os elementos do vetor média e da 
matriz de covariâncias são atualizados pelas Equações \ref{eq:ekf-update} e \ref{eq:ekf-estimate-error-covariance}, mesmo o cálculo da inovação sendo esparso. Isso ocorre porque no EKF todas as \textit{landmarks} são 
correlacionadas, mesmo que muitas dessas correlações sejam próximas de zero. 
Esse tipo de correlação ``fraca'' será explorada pelo Filtro de Informação Extendido Esparso, a fim de obter-se um algoritmo de estimação mais eficiente.

\begin{figure}[h]
  \centering
  \includestandalone[width=.45\textwidth]{figs/ekfslam-update}
  \caption{O vetor média e a matriz de covariâncias são completamente atualizados durante a observação de uma \textit{landmark}. Retirado de \cite[p.~8]{jsola}.}
  \label{fig:ekf-slam-update}
\end{figure}

\subsection{EKF-SLAM: Inserção de \textit{landmark} (aumento do vetor de estados)}
\label{sec:ekf-slam-landmark-insertion}
Nas Seções anteriores, \ref{sec:ekf-slam-prediction} e \ref{sec:ekf-slam-update}, foram tratadas as diferenças 
do EKF-SLAM para o EKF, nas já conhecidas pelo usuário comum do EKF, etapas de predição e atualização. No entanto, em EKF-SLAM uma nova operação aparece: A etapa de inserção de \textit{landmark}. Ela ocorre quando o robô observa uma \textit{landmark} que ainda não está no mapa, $\bsubvec{x}{M}$, e portanto não é possível calcular a inovação na Equação \ref{eq:ekf-slam-innovation}. Nesse caso, a nova \textit{landmark} deve ser adicionada ao vetor média e à matriz de covariância, aumentando a dimensão do sistema.

Para adicionar uma nova \textit{landmark} no vetor de estado, será definida 
a função $\mb{\sigma}(\bullet, \bullet)$, ela gera um novo vetor de estados 
que é resultado da concatenação do vetor atual, com a posição da nova 
\textit{landmark} calculada pelo modelo de medida inverso, descrito na Seção \ref{sec:inverse-measurement-model}, a partir da leitura $\bvec{y}^j$.
\begin{equation}
      \mb{\sigma}(\bsubvec{x}{t}, \bvec{y}^j) = \begin{bmatrix}
        \bsubvec{x}{t} \\
        \mb{f}(\bsubvec{x}{t}, \bvec{y}{^j})
      \end{bmatrix}
\end{equation}
Porém, não basta apenas adicionar a nova \textit{landmark} no vetor de estados, é necessário adicioná-la também na matriz de covariâncias. Quando 
o robô observa uma nova \textit{landmark}, é esperado que o erro de estimação 
da posição dessa nova \textit{landmark} seja influenciado pelo erro da 
pose do robô, no momento da leitura, e pelo erro de de medição do sensor. 

Inicializar a covariância da nova \textit{landmark} com $\infty$ (ou 
números muito grandes), como indicado em 
\cite[p.~317]{bongard2006probabilistic}, pode ser injusto. Portanto devemos 
calcular o erro, $\mb{\alpha}$, da nova estimativa do vetor aumentado, de 
maneira análoga à forma como é feita nos passos de predição e atualização 
do EKF. 
\begin{equation}
\begin{aligned}
   \mb{\alpha} &= \bsubvec{x}{t}^* - \mean{t}^*\\
   &= \begin{bmatrix}
       \bsubvec{x}{t} \\ \mb{f}(\bsubvec{x}{t}, \bvec{y})
   \end{bmatrix} - \begin{bmatrix}
       \mean{t} \\ \mb{f}(\mean{t}, \bvec{y}^j)
   \end{bmatrix} = \begin{bmatrix}
       \bsubvec{x}{t} - \mean{t}\\
       \mb{f}(\bsubvec{x}{t}, \bvec{y}) - \mb{f}(\mean{t}, \bvec{y}^j)
   \end{bmatrix}\\
   &= \begin{bmatrix}
       \mb{\eta}_t \\
       \cancel{\mb{f}(\mean{t}, \bvec{y}^j)} + 
       \bvec{F}_X \mb{\eta}_t + \bvec{F}_Y \mb{\delta}
       - \cancel{\mb{f}(\mean{t}, \bvec{y}^j)}
   \end{bmatrix} \\
    &= \begin{bmatrix}
       \mb{\eta}_t \\
       \bvec{F}_X \mb{\eta}_t + \bvec{F}_Y \mb{\delta} 
   \end{bmatrix} \text{ Onde $\mb{\eta}_t = \bsubvec{x}{t} - \mean{t}$ e $\mb{\delta} = \bvec{y} - \bvec{y}^j$}
\end{aligned}
\end{equation}
A matriz de covariância do sistema aumentado, $\cov{t}^*$, é obtida por:
\renewcommand{\arraystretch}{1.5}
\newcommand{\eqCovarianceAugmented}{\begin{bmatrix}
       \cov{t} & \cov{t} \bvec{F}_X^T\\
       \bsubvec{F}{X} \cov{t} &  \bvec{F}_X \cov{t} \bvec{F}_X^T 
       + \bvec{F}_Y \bvec{Q} \bvec{F}_Y^T
    \end{bmatrix}}
\begin{equation}
\begin{aligned}
  \cov{t}^* &= \expv{\mb{\alpha} \,\mb{\alpha}^T} \\
  &= \eqCovarianceAugmented
\end{aligned}
\label{eq:ekf-landmark-insertion-covariance}
\end{equation}
\renewcommand{\arraystretch}{1}
Portanto, a matriz de covariância do sistema aumentado é a matriz de 
covariância do sistema antes da inserção da nova \textit{landmark}, 
concatenada as covariâncias cruzadas $\cov{t} \bvec{F}_X^T$ e $\bvec{F}_X \cov{t}$, e com a covariância $\bvec{F}_X \cov{t} \bvec{F}_X^T + \bvec{F}_Y 
\bvec{Q} \bvec{F}_Y^T$, da nova \textit{landmark} inserida no mapa. 

Vale notar que o jacobiano $\bvec{F}_X$ descrito na Equação \ref{eq:inverse-measurement-model-jacobian-state-part} é esparso, logo a complexidade de $\cov{t}\bvec{F}_x$ pode ser reduzida de $\bigO{n^2}$ para $\bigO{n}$ se todos os cálculos inúteis forem ignorados, logo toda a operação de inserção de \textit{landmark} tem custo $\bigO{n}$. A Figura \ref{fig:ekf-slam-landmark-insertion} mostra o vetor média e a matriz de covariância com as novas inserções destacadas.

\begin{figure}[h]
  \centering
  \includestandalone[width=.6\textwidth]{figs/ekfslam-landmark-insertion}
  \caption{Vetor média e matriz de covariância aumentados após inserção de nova \textit{landmark}. As partes adicionadas, em cinza, correspondem às covariâncias cruzadas entre a nova landmark e o vetor de estados anterior (cinza claro), e à média da nova \textit{landmark} e sua covariância (cinza escuro). Adaptado de \cite[p.~11]{jsola}.}
  \label{fig:ekf-slam-landmark-insertion}
\end{figure}

\section{Filtro de Informação Extendido (EIF)}
O Filtro de Kalman Extendido, apresentado na Seção \ref{sec:ekf} utiliza a
parametrização de momentos (Eq. \ref{eq:moments-gaussian-probabilistic-distribution}) para representar a 
distribuição de probabilidade 
gaussiana. Já o Filtro de Informação utiliza a chamada representação canônica, composta pelo vetor de informação, $\infov{}$, e pela matriz de informação, $\infom{}$. Definidos a seguir:
\begin{equation}
  \infov{} = \cov{}^{-1} \mean{}
  \label{eq:info-vector}
\end{equation}
\begin{equation}
  \infom{} = \cov{}^{-1}
  \label{eq:info-matrix}
\end{equation}
Com as parametrizações acima, o EKF pode ser reescrito na forma do Filtro de 
Informação Extendido:
\begin{align}
  \mean{t-1} &= \infom{t-1}^{-1} \infov{t-1}
  \label{eq:eif-mean-recuperation} \\
  \predinfom{t} &= (\bsubvec{G}{t} \infom{t-1}^{-1} \bsubvecT{G}{t} + \bsubvec{R}{t})^{-1}
  \label{eq:eif-info-matrix-prediction}\\
  \predmean{t} &= \mb{g}(\mean{t-1}, \bsubvec{u}{t})
  \label{eq:eif-mean-prediction} \\
  \predinfov{t} &= \predinfom{t}\,\predmean{t}
  \label{eq:eif-info-vector-prediction} \\
  \infom{t} &= \predinfom{t} + \bsubvecT{H}{t} \bsubvec{Q}{t}^{-1} \bsubvec{H}{t}
  \label{eq:eif-info-matrix-update} \\
  \infov{t} &= \predinfov{t} + \bsubvecT{H}{t}\bsubvec{Q}{t}^{-1} 
  \brac{\bsubvec{y}{t} - \mb{h}(\predmean{t}) + \bsubvec{H}{t}\predmean{t}}
  \label{eq:eif-info-vector-update}
\end{align}
e a distribuição de probabilidade na Eq. \ref{eq:moments-gaussian-probabilistic-distribution} pode ser reescrita como:
\begin{equation}
  \pr{\bvec{x}} = \boldsymbol{\eta} \exp\parentheses{-\frac{1}{2} 
  \bvecT{x} \infom{} \bvec{x} + \bvecT{x}\infov{}}
  \label{eq:canonical-gaussian-probabilistic-distribution}
\end{equation}

Uma vantagem do Filtro de Informação é que ele tende a ser numericamente mais estável. Além disso, 
representar alto nível de incerteza é numericamente mais seguro quando comparado 
com o Filtro de Kalman, aqui basta definir $\infom{} = \bvec{0}$, enquanto no KF 
é necessário utilizar valores muito grandes na matriz de covariância. Outro 
aspecto interessante do IF é sua naturalidade para sistemas multi robôs, onde a 
informação é coletada de maneira descentralizada \cite[p.~78]{bongard2006probabilistic}.

Porém, as principais desvantagens do EIF são a necessidade da recuperação da média em \ref{eq:eif-mean-recuperation} e, a predição da matriz de informação em 
\ref{eq:eif-info-matrix-prediction}, pois ambas operações envolvem a inversão 
da matriz de informação, cuja complexidade é $\bigO{n^3}$. Embora, no EKF 
também seja 
necessário inverter a matriz de covariância da inovação, $\bvec{Z}$, 
essa usualmente possui dimensão menor que a matriz de informação. 
Em geral, para sistemas de grande dimensão acredita-se que o EIF seja 
computacionalmente inferior, do ponto de vista de tempo de execução, em relação 
ao EKF. Por esse motivo ele é menos utilizado que o EKF, na prática 
\cite[p.~78]{bongard2006probabilistic}.

No entanto, ao empregar o EIF no problema SLAM nota-se que grande parte dos 
blocos fora da diagonal principal da matriz de informação são quase nulos, ou 
seja, agregam pouca informação ao sistema. Isso se deve à estrutura do problema 
SLAM, pois grande parte das correlações entre \textit{landmarks} (portanto 
fora da diagonal principal) são propagadas pela incerteza na pose do robô, 
quando essas (as \textit{landmarks}) são observadas por ele. Apenas 
\textit{landmarks} dentro de uma mesma vizinhança são observadas juntas 
resultando em alta correlação.

Esse aspecto é explorado pelo Filtro de Informação Extendido Esparso (SEIF), 
por meio de aproximações o SEIF mantém a matriz de informação diagonalizada, 
aproximando os elementos fora da diagonal para zero. Isso leva o SEIF a otimizar 
operações e ter complexidade de tempo constante, enquanto mantém uso linear de 
memória. A próxima Seção descreve o SEIF e seus detalhes de implementação.

\section{SEIF-SLAM}
Essa Seção descreve o Filtro de Informação Extendido Esparso (SEIF), e 
como ele endereça as principais desvantagens do EIF clássico no 
contexto de SLAM. Será mostrado como ele mantém complexidade 
linear no uso de memória, e complexidade de tempo constante nos passos de predição e atualização, independentemente do número de 
\textit{landmarks} no mapa/ambiente.

Para atingir essas façanhas, o SEIF mantém a matriz de informação 
com formato próximo ao de uma matriz diagonal, por meio do uso de 
\textit{landmarks} ativas e passivas, que serão descritas mais adiante. 
Além disso, a recuperação da média, na Equação 
\ref{eq:eif-mean-recuperation}, é modelada como um problema de 
otimização. As Seções a seguir são baseadas na discussão em \cite[Capítulo~12.4]{bongard2006probabilistic}.

\subsection{SEIF-SLAM: Landmarks ativas e passivas}
A diferença fundamental entre o SEIF-SLAM e o EIF-SLAM está na estrutura da matriz de informação, no SEIF ela é esparsa, ou melhor, 
\emph{esparsificada}. Enquanto no EKF-SLAM/EIF-SLAM temos 
que $\covf{\bvec{m}^j, \bvec{m}^k} \neq \bvec{0}, 
\forall\, \{j,k\}$, ou seja, que as posições de todas as 
\textit{landmarks} são correlacionadas, o SEIF tenta eliminar a maioria 
dessas correlações, a fim de obter uma matriz de informação esparsa.

Para isso, ele mantém dois conjuntos de landmarks $\bsubvec{m}{t}^+$ e $\bsubvec{m}{t}^-$, cuja inter relação está descrita na Equação \ref{eq:seif-active-passive-sets}. O conjunto $\bsubvec{m}{t}^+$ é composto pelas \textit{landmarks} ativas, que estão
``ligadas'' ao robô no tempo $t$, ou seja, 
$\covf{\bsubvec{x}{R, t}, \bsubvec{m}{t}^+} \neq 0$. Já o conjunto 
$\bsubvec{m}{t}^-$ é 
formado pelas \textit{landmarks} passivas, que não estão correlacionadas com 
a pose atual do robô, ou seja, 
$\covf{\bsubvec{x}{R, t}, \bsubvec{m}{t}^-} = 0$.
\begin{equation}
\begin{cases}
  \bsubvec{m}{t}^+ \cup \bsubvec{m}{t}^- &= \bsubvec{x}{M} \\
  \bsubvec{m}{t}^+ \cap \bsubvec{m}{t}^- &= \emptyset
\end{cases}
\label{eq:seif-active-passive-sets}
\end{equation}
Uma das consequências desse esquema, é que as \textit{landmarks} não são 
globalmente correlacionadas entre sí, como ocorre no EKF-SLAM e EIF-SLAM. Na 
verdade, aqui, elas são localmente correlacionadas com sua vizinhança. Onde 
vizinhança é definida como o conjunto de \textit{landmarks} presentes em 
$\bsubvec{m}{t}^+$ concomitantemente. Portanto, a inovação de uma 
\textit{landmark} observada afeta apenas a pose do robô e de sua vizinhança, 
ao contrário do que acontece no EKF/EIF onde a inovação de uma 
\textit{landmark} afeta todo o sistema.

O conjunto $\bsubvec{m}{t}^+$ contém as $k$ últimas \textit{landmarks} 
observadas até o instante $t$, onde $k$ é o tamanho do conjunto. As 
\textit{landmarks} vão entrando e saindo desse conjunto conforme o robô 
navega no ambiente e novas \textit{landmarks} vão sendo observadas enquanto 
outras deixam, de sê-lo.

Nas próximas Seções, ficará claro que o tamanho definido para o conjunto de 
landmarks ativas limitará a quantidade de elementos longe da diagonal 
principal da matriz de informação, tornando-a esparsa. É essa 
característica que confere ao SEIF-SLAM a complexidade linear em memória e 
tempo constante de atualização e predição.

\subsection{SEIF-SLAM: Passo de predição}
\newcommand{\psitvalue}{\bsubvecT{M}{x_r} \parentheses{\bsubvec{G}{R, t}^{-1} - \bsubvec{I}{3}} \bsubvec{M}{x_r}}
\newcommand{\kappatvalue}{\bsubvec{\Phi}{t} - \bsubvec{\Phi}{t}\bsubvecT{M}{x_r}
    \parentheses{\bsubvec{R}{R,t}^{-1} + \bsubvec{M}{x_r}\bsubvec{\Phi}{t}
      \bsubvecT{M}{x_r}}^{-1} \bsubvec{M}{x_r}\bsubvec{\Phi}{t}
}
\newcommand{\lambdatvalue}{\bsubvecT{\Psi}{t}\infom{t-1} + \bsubvecT{\Psi}{t}\infom{t-1}\bsubvec{\Psi}{t} + \infom{t-1}\bsubvec{\Psi}{t}}
\newcommand{\predXiTValue}{\parentheses{\mb{\lambda}_t - \mb{\kappa}_t}\mean{t-1} + \infov{t-1} + \predinfom{t}\bsubvecT{M}{x_r} \mb{\delta}_{r,t}}
\newcommand{\predmeanTValue}{\mean{t-1} + \bsubvecT{M}{x_r} \mb{\delta}_{r,t}}
O passo de predição do SEIF-SLAM está condensado no Algoritmo \ref{alg:seif-slam-prediction}, abaixo. As Seções que se seguem derivam os 
passos do algoritmo a partir das equações de predição do EIF, \ref{eq:eif-info-matrix-prediction}, \ref{eq:eif-info-vector-prediction} e \ref{eq:eif-mean-prediction}. A esparsidade da matriz de informação é 
usada como premissa para garantir o tempo de execução constante, 
a esparsificação em sí será tratada mais adiante, por hora vamos assumir que a matriz é esparsa.

TODO: 
\begin{itemize}
  \item \blue{nota sobre a cópia da matriz de informação}
  \item \blue{nota sobre correção da componente de orientação}
\end{itemize}
\begin{algorithm}[h]
  \setstretch{1.4}
  \caption{SEIF-SLAM passo de predição}
  \label{alg:seif-slam-prediction}
  \begin{algorithmic}[1]
    \Procedure{SEIF-SLAM-Prediction}{$\infov{t-1}, \mean{t-1}, \infom{t-1}, \bsubvec{u}{t}$} 
    \State $\bsubvec{\Psi}{t} \gets \psitvalue $
    \State $\mb{\lambda}_t \gets \lambdatvalue$
    \State $\bsubvec{\Phi}{t} \gets \infom{t-1} + \mb{\lambda}_t$
    \State $\mb{\kappa}_t \gets \kappatvalue$
    \State $\predinfom{t} \gets \bsubvec{\Phi}{t} - \mb{\kappa}_t$
    \State $\predinfov{t} \gets \predXiTValue$
    \State $\predmean{t} \gets \predmeanTValue$
    \State \Return $\predinfov{t}, \predmean{t}, \predinfom{t}$
    \EndProcedure
  \end{algorithmic}
\end{algorithm}

Antes é importante relembrar que o jacobiano do sistema SLAM tem a seguinte 
forma:
\begin{equation*}
  \slamsystemjacobian \tag{\ref{eq:slam-system-jacobian} repetida}
\end{equation*}
Além disso, vamos definir o ruído do modelo do sistema, $\bsubvec{R}{t}$, como:
\begin{equation}
\begin{aligned}
\bsubvec{R}{t} &= \begin{bmatrix}
  \bsubvec{R}{R, t} & \bvec{0} \\
  \bvec{0} & \bvec{0} \\
  \end{bmatrix}\\
  &= \begin{bmatrix}
    1 & 0 & 0 & 0 \cdots 0 \\
    0 & 1 & 0 & 0 \cdots 0 \\
    0 & 0 & 1 & 0 \cdots 0
  \end{bmatrix}^T \bsubvec{R}{R, t}
  \begin{bmatrix}
    1 & 0 & 0 & 0 \cdots 0 \\
    0 & 1 & 0 & 0 \cdots 0 \\
    0 & 0 & 1 & 0 \cdots 0
  \end{bmatrix} \\
  &= \bsubvec{M}{x_r}^T \, \bsubvec{R}{R, t} \, \bsubvec{M}{x_r}
\end{aligned}
\label{eq:slam-system-error}
\end{equation}

\subsubsection{Predição da Matriz de Informação}
Primeiro, vamos reescrever a Equação \ref{eq:eif-info-matrix-prediction} em 
termos de $\bsubvec{\Phi}{t}$ e, $\bsubvec{R}{R,t}$:
\begin{equation}
  \predinfom{t} = (\bsubvec{\Phi}{t}^{-1} + 
    \bsubvec{M}{x_r}^T \bsubvec{R}{R, t}\bsubvec{M}{x_r})^{-1}
  \label{eq:seif-info-matrix-prediction}
\end{equation}
Onde:
\newcommand{\seifPhi}{\brac{\bsubvecT{G}{t}}^{-1} \infom{t-1} \brac{\bsubvec{G}{t}}^{-1}}
\begin{equation}
\begin{aligned}
  \bsubvec{\Phi}{t} &= (\bsubvec{G}{t} \infom{t-1}^{-1} \bsubvecT{G}{t})^{-1}\\
  &= \seifPhi \\
  \label{eq:seif-phi}
\end{aligned}
\end{equation}
Aplicando o lema da inversão (Apêndice \ref{app:inversion-lemma}) em \ref{eq:seif-info-matrix-prediction}, temos:
\begin{equation}
\begin{aligned}
  \predinfom{t} &= \bsubvec{\Phi}{t} - \bsubvec{\Phi}{t}\bsubvecT{M}{x_r}
    \parentheses{\bsubvec{R}{R,t}^{-1} + \bsubvec{M}{x_r}\bsubvec{\Phi}{t}
      \bsubvecT{M}{x_r}}^{-1} \bsubvec{M}{x_r}\bsubvec{\Phi}{t} \\
  &= \bsubvec{\Phi}{t} - \mb{\kappa}_t
\end{aligned}
\end{equation}
Para calcularmos
\begin{equation}
  \mb{\kappa}_t = \kappatvalue
  \label{eq:seif-slam-kappa}
\end{equation}
em tempo constante, temos que calcular $\bsubvec{\Phi}{t}$ em tempo constate 
a partir de $\infom{t-1}$. 
Para isso, vamos representar $\bsubvec{G}{t}$ como na Equação \ref{eq:slam-system-jacobian}, e calcular sua inversa:
\begin{equation}
\begin{aligned}
  \bsubvec{G}{t}^{-1} &= \begin{bmatrix}
    \bsubvec{G}{R, \,t} & \bvec{0} \\
    \bvec{0} & \bvec{I}
  \end{bmatrix}^{-1} \\
  &= \begin{bmatrix}
    \bsubvec{G}{R, \,t}^{-1} & \bvec{0} \\
    \bvec{0} & \bvec{I} 
  \end{bmatrix} & \text{(inversão de matriz bloco diagonal)} \\
  &= \bsubvec{I}{n} + \begin{bmatrix}
    \bsubvec{G}{R, \,t}^{-1} - \bsubvec{I}{3} & \bvec{0} \\
    \bvec{0} & \bvec{0}
  \end{bmatrix} \\
  &= \bsubvec{I}{n} + \overbrace{\psitvalue}^{\bsubvec{\Psi}{t}} \\
  &= \bvec{I} + \bsubvec{\Psi}{t} 
\end{aligned}
\label{eq:seif-slam-psi}
\end{equation}
Note que $\bsubvec{\Psi}{t}$ é uma matriz de dimensão $n$, onde apenas os 
elementos do bloco superior esquerdo $3\times 3$ , são diferentes de zero. Usando a Equação \ref{eq:seif-slam-psi} na Equação \ref{eq:seif-phi} temos:
\begin{equation}
\begin{aligned}
  \bsubvec{\Phi}{t} &= \seifPhi \\
  &= \parentheses{\bvec{I} + \bsubvecT{\Psi}{t}} \infom{t-1} \parentheses{\bvec{I} + \bsubvec{\Psi}{t}} \\
  &= \infom{t-1} + \underbrace{\lambdatvalue}_{\mb{\lambda}_t} \\
  &= \infom{t-1} + \mb{\lambda}_t
\end{aligned}
\end{equation}
Como $\bsubvec{\Psi}{t}$ é esparsa e com quantidade de elementos não nulos 
constante, $\mb{\lambda}_t$ também será esparsa e com elementos não nulos 
constantes, pois, ambas dependem apenas do modelo de movimento do robô e das 
covariâncias cruzadas da posição do robô com as posições das 
\textit{landmarks} ativas, que são quantidades constantes.

Portanto, o cálculo de $\bsubvec{\Phi}{t}$ a partir de $\infom{t-1}$
é constante, pois resulta da subtração dos elementos não nulos 
de $\mb{\lambda}_t$, de $\infom{t-1}$. Na Figura 
\ref{fig:seif-slam-prediction-variables} são representadas as matrizes $\bsubvec{\Psi}{t}, \mb{\lambda}_t, \mb{\kappa}_t$ obtidas durante a 
execução do SEIF-SLAM com tamanho do conjunto de \textit{landmarks} ativas 
igual a dois. 
\begin{figure}[h]
  \begin{subfigure}{.30\textwidth}
    \includestandalone[width=\textwidth]{figs/seifslam-psi}
    \caption{$\bsubvec{\Psi}{t}$}
  \end{subfigure}
  \hfill
  \begin{subfigure}{.30\textwidth}
    \includestandalone[width=\textwidth]{figs/seifslam-lambda}
    \caption{$\mb{\lambda}_t$}
  \end{subfigure}
  \hfill
  \begin{subfigure}{.30\textwidth}
    \includestandalone[width=\textwidth]{figs/seifslam-kappa}
    \caption{$\mb{\kappa}_{t}$}
  \end{subfigure}
  \caption{Representação das matrizes $\bsubvec{\Psi}{t}, \mb{\lambda}_t, \mb{\kappa}_t$, necessárias para calcular a 
  matriz de informação predita durante o movimento do robô. Os elementos 
  nulos são representados em branco, e os não nulos em cinza. 
  As matrizes acima 
  pertencem a um sistema SEIF-SLAM de um robô diferencial e 
  sensor laser do tipo LiDAR, com duas \textit{landmarks} ativas, ou seja, $\lvert \bvec{m}^+ \rvert = 2$. A quantidade de elementos não nulos 
  é uma constante dada em função do modelo de movimento do robô e do tamanho do conjunto $\bvec{m}^+$, independentemente do tamanho do mapa. Neste momento a terceira e quinta \textit{landmark} estavam ativas.}
  \label{fig:seif-slam-prediction-variables}
\end{figure}

A princípio, pode parecer que a quantidade de elementos não 
nulos é significativa em relação ao tamanho das matrizes. Porém, essa 
percepção se deve ao fato das matrizes representadas na Figura \ref{fig:seif-slam-prediction-variables} serem pequenas, o tamanho delas foi escolhido de modo a facilitar a visualização.

\subsubsection{Predição do vetor de informação}
Abaixo é apresentada uma série de manipulações para que a predição do vetor 
de informação, na Equação \ref{eq:eif-info-vector-prediction}, possa ser 
realizada em tempo constante no SEIF-SLAM. Primeiro, vamos reescrever o 
modelo de movimento na Equação \refname{eq:motion-model}, como:
\begin{equation}
  \bsubvec{x}{t} = \bsubvec{x}{t-1} + \mb{\delta}_t
  \label{eq:motion-model-rewritten}
\end{equation}
Partindo da Equação \ref{eq:eif-info-vector-prediction} e utilizando \ref{eq:motion-model-rewritten} acima, temos:
\begin{equation}
  \begin{aligned}
    \predinfov{t} &= \predinfom{t}\, \predmean{t} \\
    &= \predinfom{t}\parentheses{\predmeanTValue} \\
    &= \predinfom{t}\parentheses{\infom{t-1}^{-1}\infov{t-1} + \bsubvecT{M}{x_r} \mb{\delta}_{r,t}} \\
    &= \predinfom{t} \infom{t-1}^{-1}\infov{t-1} + \predinfom{t}\bsubvecT{M}{x_r} \mb{\delta}_{r,t} \\
    &= \parentheses{\predinfom{t} + \underbrace{\infom{t-1} - \infom{t-1}}_{\bvec{0}} + \underbrace{\bsubvec{\Phi}{t} - \bsubvec{\Phi}{t}}_\bvec{0}} \infom{t-1}^{-1}\infov{t-1} + \predinfom{t}\bsubvecT{M}{x_r} \mb{\delta}_{r,t} \\
    &= \parentheses{\underbrace{\predinfom{t}- \bsubvec{\Phi}{t}}_{-\mb{\kappa}_t} + \infom{t-1} + \underbrace{\bsubvec{\Phi}{t} - \infom{t-1}}_{\mb{\lambda}_t} } \infom{t-1}^{-1}\infov{t-1} + \predinfom{t}\bsubvecT{M}{x_r} \mb{\delta}_{r,t} \\
    &= \parentheses{\mb{\lambda}_t - \mb{\kappa}_t + \infom{t-1}} \infom{t-1}^{-1}\infov{t-1} + \predinfom{t}\bsubvecT{M}{x_r} \mb{\delta}_{r,t} \\
    &= \parentheses{\mb{\lambda}_t - \mb{\kappa}_t}\infom{t-1}^{-1}\infov{t-1} + \infom{t-1}\infom{t-1}^{-1}\infov{t-1} + \predinfom{t}\bsubvecT{M}{x_r} \mb{\delta}_{r,t} \\
    &=  \predXiTValue \\
  \end{aligned} 
\end{equation}
Como $\mb{\lambda}_t$ e $\mb{\kappa}_t$ são ambas esparsas, o produto $\parentheses{\mb{\lambda}_t - \mb{\kappa}_t}\mean{t-1}$ contém um número 
determinado de elementos não nulos, e portanto é calculado em tempo 
constante. O produto $\parentheses{\mb{\lambda}_t - \mb{\kappa}_t}\mean{t-1}$ 
resulta em uma matriz nula exceto pelo primeiro bloco $3\times 3$, e ao 
multiplica-la pela matriz de informação predita, que também é esparsa, 
temos como resultado um vetor esparso 
\cite[p.~398]{bongard2006probabilistic}. Portanto, é necessário um número 
constante de operações, que independe do tamanho do mapa, para calcular o 
vetor de informação predito.

\subsection{SEIF-SLAM: Recuperação da média}
Outro desafio a ser resolvido no SEIF-SLAM, para manter o tempo de 
predição e/ou atualização constantes, é o cálculo do vetor média. 
Fundamental para as linearizações dos modelos de movimento e medida, e para no cálculo do vetor de informação na etapa de esparsificação. Na Equação 
\ref{eq:eif-mean-recuperation}, do Filtro de Informação Extendido, esse cálculo 
é feito através da inversão da matriz de informação, 
e, mesmo numa matriz esparsa, essa operação não é constante.

Para contornar essa inversão, no SEIF esse passo é executado de uma forma completamente diferente: ele é modelado como um problema de otimização. Em 
\cite[Cap.~12.6]{bongard2006probabilistic} mostra-se que o vetor média equivale 
ao ponto de máximo da quadrática na representação canônica da gaussiana 
Eq. \ref{eq:canonical-gaussian-probabilistic-distribution}, então o 
vetor média atualizado $\mean{t}$ pode ser calculado como:
\begin{equation}
  \begin{aligned}
    \mean{t} &= \mathop{\mathrm{argmax}}\limits_{\mean{}} \,\exp\parentheses{-\frac{1}{2}\,\mean{}^T\infom{t}\mean{} 
    + \mean{}^T \infov{t}}\\
    &= \mathop{\mathrm{argmin}}\limits_{\mean{}}\, - \parentheses{-\frac{1}{2}\,\mean{}^T\infom{t}\mean{} 
    + \mean{}^T \infov{t}}\\
    &= \mathop{\mathrm{argmin}}\limits_{\mean{}}\, \frac{1}{2}\,\mean{}^T\infom{t}\mean{} 
    - \mean{}^T \infov{t}
  \end{aligned}
  \label{eq:seif-mean-recuperation}
\end{equation}

A minimização acima pode ser realizada por qualquer algoritmo de 
minimização como gradiente descendente, gradiente conjugado, entre 
outros. Nesse trabalho foi utilizado a descida de coordenadas, Algoritmo \ref{alg:seif-slam-mean-recovery}, que apesar de não ser tão sofisticada 
quanto os algoritmos citados, se mostrou boa o suficiente para a tarefa.

Em problemas de otimização um bom palpite inicial próximo da vizinhança 
do ponto de interesse é fundamental para a convergência do algoritmo, no 
caso de SLAM esperamos que o estado atualizado sempre esteja próximo do 
estado predito, portanto o vetor $\predmean{t}$ é utilizado como ponto de partida da otimização.
\begin{algorithm}[h]
  \setstretch{1.25}
  \caption{Etapa de recuperação da média no SEIF-SLAM}
  \label{alg:seif-slam-mean-recovery}
\begin{algorithmic}[1]
  \Procedure{SEIF-SLAM-Recuperação-Média}{$\predmean{t}, \infov{t}, \infom{t}, \mathrm{K}$}
  \State $\mean{t} \gets \predmean{t}$
  \For {$j \in {\bvec{m}^+}$}
    \State $k \gets 0$
    \While{$k < \mathrm{K}$ }
      \State $\mean{t}^{m^j} \gets (\bsubvec{M}{m^j} \infom{t} \bsubvecT{M}{m^j})^{-1} \, \bsubvec{M}{m^j}\brac{\infov{t} - 
      \infom{t}\predmean{t} + \infom{t} \bsubvecT{M}{m^j}
      \bsubvec{M}{m^j}\predmean{t}}$
      \State $\mean{t}^{x_r} \gets (\bsubvec{M}{x_r} \infom{t} \bsubvecT{M}{x_r})^{-1} \, \bsubvec{M}{x_r}\brac{\infov{t} - 
      \infom{t}\predmean{t} + \infom{t} \bsubvecT{M}{x_r}
      \bsubvec{M}{x_r}\predmean{t}}$
      \State $k \gets k + 1$
    \EndWhile
  \EndFor
  \State \Return $\mean{t}$
  \EndProcedure
\end{algorithmic}
\end{algorithm}

Como pode ser notado na linha 2 do algoritmo acima, dentre todas as 
\textit{landmarks} apenas as ativas são atualizadas. Além da pose do 
robô, é claro. Isso se deve novamente à esparsidade da matriz de 
informação, pois a atualização de uma \textit{landmark} só contribui 
para a estimação da posição de suas vizinhas e da pose do robô. 
Portanto, a complexidade da recuperação da média também é constante, 
dado que $\mathrm{K}$ e o tamanho do conjunto $\bvec{m}^+$ são ambos 
constantes.

\subsection{SEIF-SLAM: Passo de atualização}
Por hora, assim como foi feito com o EKF-SLAM na Eq. \ref{eq:ekf-slam-innovation} vamos assumir que é possível calcular a inovação 
$\bsubvec{z}{t}^j$ a partir da medida $\bsubvec{y}{t}^j$. É claro que 
na prática, é preciso antes uma etapa que estabeleça correspondência entre 
as medidas lidas e as \textit{landmarks} existentes no vetor de estados. 
Essa correspondência é fundamental para utilizar a \textit{landmark} $\bvec{m}^i$ correta no cálculo do modelo de medida \measurementModel, Eq. \ref{eq:range-bearing-measurement-model}.

O algoritmo abaixo pressupõe que tais correspondências são 
conhecidas. Mais adiante na Seção \ref{sec:data-association} será 
abordado uma maneira de como essas correspondências são estabelecidas. 
Ao contrário dos outros passos, no passo de atualização a esparsidade da 
matriz de informação não influencia e/ou altera nada em relação ao 
Filtro de Informação Extendido. As equações de atualização nas linhas 
3 e 4 abaixo, são as mesmas equações \ref{eq:eif-info-matrix-update} e 
\ref{eq:eif-info-vector-update} do EIF.

\begin{algorithm}[h]
  \setstretch{1.25}
  \caption{Etapa de atualização do SEIF-SLAM com associação conhecida}
  \label{alg:seif-slam-update-known-associations}
\begin{algorithmic}[1]
  \Procedure{SEIF-SLAM-Atualização}{$\predmean{t}, \predinfov{t}, \predinfom{t}, \bvec{y}, j = \text{índice da \textit{landmark}}$}
  \State $\bvec{z} \gets \bvec{y} - \mb{h}^j(\predmean{t})$
  \State $\infom{t} \gets \predinfom{t} + \bsubvecT{H}{t} \bsubvec{Q}{t}^{-1} \bsubvec{H}{t}$
  \State $\infov{t} \gets \predinfov{t} + \bsubvecT{H}{t}\bsubvec{Q}{t}^{-1} 
  \brac{\bvec{z} + \bsubvec{H}{t}\predmean{t}}$
  \State $\mean{t} \gets \textproc{SEIF-SLAM-Recuperação-Média}(\predmean{t}, \infov{t}, \infom{t}, \mathrm{K})$
  \State \Return $\infov{t}, \mean{t}, \infom{t}$
  \EndProcedure
\end{algorithmic}
\end{algorithm}

A atualização acima altera apenas os elementos referentes ao robô e à 
j-ésima \textit{landmark} tanto na matriz de informação quanto no vetor 
de informação é ilustrada na Figura \ref{fig:seif-slam-update}. O SEIF 
herda o caráter "local" da atualização do EIF, e difere no EKF onde a 
atualização altera todos os elementos do vetor média e da matriz de 
covariâncias como foi mostrado na Figura \ref{fig:ekf-slam-update}.
\begin{figure}[h]
  \begin{subfigure}{.30\textwidth}
    \includestandalone[width=\textwidth, keepaspectratio]{figs/seifslam-mean-update}
    \caption{$\mean{t}$}
  \end{subfigure}
  \hfill
  \begin{subfigure}{.30\textwidth}
    \includestandalone[width=\textwidth]{figs/seifslam-infov-update}
    \caption{$\infov{t}$}
  \end{subfigure}
  \hfill
  \begin{subfigure}{.3\textwidth}
    \includestandalone[width=\textwidth]{figs/seifslam-infom-update}
    \caption{$\infom{t}$}
  \end{subfigure}
  \caption{Representação dos vetores média e informação, $\mean{t}$ e $\infov{t}$ respectivamente, e da matriz de informação $\infom{t}$ na 
  etapa de atualização. O conjunto $\bvec{m}^+$ possui tamanho 2, e a 
  segunda \textit{landmark} é atualizada. Note que no vetor média, 
  a pose do robô e todas as \textit{landmarks} ativas são atualizadas, 
  enquanto que no vetor e matriz de informação, apenas a pose do robô e 
  a posição da \textit{landmark} observada são atualizados.}
  \label{fig:seif-slam-update}
\end{figure}

\subsection{SEIF-SLAM: Inserção de nova \textit{landmark}}
Para derivar a inserção de \textit{landmark} no EIF/SEIF, vamos partir 
dos resultados da inserção de \textit{landmark} no EKF apresentados na 
Seção \ref{sec:ekf-slam-landmark-insertion}. Utilizando o resultado do 
aumento da matriz de covariâncias, repetido a seguir:
\begingroup
\renewcommand{\arraystretch}{1.5}
\begin{equation*}
  \bsubvec{P}{t}^* = \eqCovarianceAugmented \tag{\ref{eq:ekf-landmark-insertion-covariance} repetida}
\end{equation*}
\endgroup
E aplicando a equivalência $\bvec{\Omega}^{-1} = \bvec{P}$, pode-se reescrever a expressão acima como:
\newcommand{\eqOmegaInv}{\bsubvec{\Omega}{t}^{-1}}
\begingroup
\renewcommand{\arraystretch}{1.5}
\begin{equation}
  \brac{\bsubvec{\Omega}{t}^*}^{-1} = \begin{bmatrix}
    \eqOmegaInv & \eqOmegaInv \bsubvecT{F}{X} \\
    \bsubvec{F}{X} \eqOmegaInv & \bsubvec{F}{X}\eqOmegaInv\bsubvecT{F}{X} + \bsubvec{F}{Y} \bvec{Q} \bsubvecT{F}{Y}
  \end{bmatrix}
\end{equation}
Invertendo ambos os lados:
\begin{equation*}
  \bsubvec{\Omega}{t}^* = \begin{bmatrix}
    \bvec{X} & \bvec{Y} \\
    \bvec{Z} & \bvec{U}
  \end{bmatrix}
\end{equation*}
\endgroup
Onde, pelo lema da inversão na forma de blocos (Apêndice \ref{app:blockwise-inversion-lemma}) temos:
\newcommand{\eqU}{\brac{\bsubvecT{F}{Y}}^{-1} \bvec{Q}^{-1} \bsubvec{F}{Y}^{-1}}
\addtolength{\jot}{1.4pt}
\begingroup
\begin{align}
  &\begin{aligned}
    \blue{\bvec{U}} &=  \brac{\bsubvec{F}{X}\eqOmegaInv\bsubvecT{F}{X} + \bsubvec{F}{Y} \bvec{Q} \bsubvecT{F}{Y} - \bsubvec{F}{X} \cancel{\eqOmegaInv \brac{\eqOmegaInv}^{-1}} \,\eqOmegaInv \bsubvecT{F}{X}}^{-1}\\
    &= \brac{\cancel{\bsubvec{F}{X}\eqOmegaInv\bsubvecT{F}{X}} + \bsubvec{F}{Y} \bvec{Q} \bsubvecT{F}{Y} - \cancel{\bsubvec{F}{X} \eqOmegaInv \bsubvecT{F}{X}}}^{-1}\\
    &= \eqU
  \end{aligned}\\
  &\begin{aligned}
    \bvec{Y} &= -\cancel{\brac{\eqOmegaInv}^{-1} \eqOmegaInv}\bsubvecT{F}{X} \blue{\bvec{U}}\\
    &= \bsubvecT{F}{X} \parentheses{\eqU} \\
  \end{aligned}\\
  &\begin{aligned}
    \bvec{Z} &= -\blue{\bvec{U}} \bsubvec{F}{X} \cancel{\eqOmegaInv \brac{\eqOmegaInv}^{-1}}\\
    &= -\parentheses{\eqU} \bsubvec{F}{X}
  \end{aligned}\\
  &\begin{aligned}
    \bvec{X} &= \brac{\eqOmegaInv}^{-1} + \cancel{\brac{\eqOmegaInv}^{-1} \eqOmegaInv}\bsubvecT{F}{X} \,\blue{\bvec{U}} \, \bsubvec{F}{X} \cancel{\eqOmegaInv \brac{\eqOmegaInv}^{-1}}\\
    &= \bsubvec{\Omega}{t} + \bsubvecT{F}{X} \parentheses{\eqU} \bsubvec{F}{X}
  \end{aligned}
\end{align}
Por fim, temos que a matriz de informação aumentada é dada por:
\renewcommand{\arraystretch}{1.5}
\begin{equation}
  \bsubvec{\Omega}{t}^* = \begin{bmatrix}
    \bsubvec{\Omega}{t} + \bsubvecT{F}{X} \parentheses{\eqU} \bsubvec{F}{X} & -\bsubvecT{F}{X} \parentheses{\eqU} \\
    -\parentheses{\eqU} \bsubvec{F}{X} & \eqU
  \end{bmatrix} 
  \label{eq:seif-landmark-insertion-information-matrix}
\end{equation}
\endgroup

Note que ao contrário da matriz de covariâncias aumentada do EKF,  na Eq. \ref{eq:ekf-landmark-insertion-covariance}, que não altera a covariância dos elementos já existentes no filtro. A matriz de informação aumentada altera a informação dos elementos existentes, mais especificamente ela altera a informação do robô.

Isso pode ser concluído a partir da observação da forma da matrix $\bsubvec{F}{X}$, Eq. \ref{eq:inverse-measurement-model-jacobian-state-part}, nula em todos elementos exceto pelo primeiro bloco $2\times 3$. Para deixar isso em evidência a Eq. \ref{eq:seif-landmark-insertion-information-matrix} é 
reescrita abaixo:
\begingroup
\renewcommand{\arraystretch}{1.5}
\begin{equation}
  \bsubvec{\Omega}{t}^* = \begin{bmatrix}
    \bsubvec{\Omega}{RR,t} + \bsubvecT{F}{R}\parentheses{\eqU}\bsubvec{F}{R} & \bsubvec{\Omega}{RM,t} & -\bsubvecT{F}{R} \parentheses{\eqU}\\
    \bsubvecT{\Omega}{RM,t} & \bsubvec{\Omega}{MM,t} & \nullmat{n-3}{2}\\
    -\parentheses{\eqU} \bsubvec{F}{R} & \nullmat{2}{n-3} & \eqU
  \end{bmatrix}
\end{equation}
\endgroup
A partir da matriz de informação e do vetor de estado aumentados pode-se 
calcular o vetor de informação aumentado:
\begingroup
\renewcommand{\arraystretch}{1.5}
\begin{equation}
  \begin{aligned}
    \infov{t}^* &= \bsubvec{\Omega}{t}^* \mean{t}^*\\
    &= \begin{bmatrix}
      \begin{bmatrix} \bsubvec{\Omega}{RR,t}^* & \bsubvec{\Omega}{RM,t}^* \end{bmatrix} \mean{t}^*  \\
      \infov{M, t}\\
      \begin{bmatrix} \bsubvec{\Omega}{RM_j,t}^* & \bsubvec{\Omega}{M_j M_j, t}^* \end{bmatrix} \mean{t}^*
    \end{bmatrix}
  \end{aligned}
\end{equation}
\endgroup
Assim como na matriz de informação, os elementos do vetor de informação correspondentes ao robô também são alterados, além dos elementos da nova \textit{landmark}, é claro. De novo, ao contrário do que acontece no EKF, onde apenas os elementos da nova \textit{landmark} são alterados no vetor de estados. A Figura \ref{fig:seif-slam-landmark-insertion} representa as porções alteradas e/ou adicionadas à matriz de informação e ao vetor de informação.
\begin{figure}[h]
  \centering
  \includestandalone[width=.6\textwidth]{figs/seifslam-landmark-insertion}
  \caption{Vetor e matriz de informação aumentados após inserção de nova \textit{landmark}. As partes adicionadas, em cinza, correspondem à informação cruzada entre a nova landmark e o robô (cinza claro), e à média da nova \textit{landmark} e do robô, e suas informações (cinza escuro).}
  \label{fig:seif-slam-landmark-insertion}
\end{figure}

Como pode ser observado na figura acima, a operação de inserção de \textit{landmark} no SEIF possui complexidade constante $\bigO{1}$ em memória (o EKF possui complexidade linear), porém assim como no EKF o cálculo das 
novas porções possui complexidade $\bigO{n}$.

\blue{\textbf{TODO}: Como interpretar  o resultado de uma nova landmark alterar a informação do robô?}

\subsection{SEIF-SLAM: Esparsificação da matriz de informação}



\section{Associação de \textit{landmarks}}
\label{sec:data-association}

\section{Estratégia para mitigar efeito de falsas detecções}

\section{Conclusão do capítulo}