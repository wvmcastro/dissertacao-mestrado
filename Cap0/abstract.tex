The Simultaneous Localization and Mapping problem, known by the acronym SLAM, asks whether it is possible for a robot to be placed in an environment
unknown a priori, and incrementally build a map of this
environment while simultaneously locating itself on this map without the
need for a location infrastructure such as GPS.

Solving the SLAM problem is critical for mobile robotics
autonomy. However, although already solved, it is not a trivial task.
both from a theoretical and implementation point of views.
Depending on robot dynamics, sensors used, computing power available or need for navigation and guidance, the solution
can become more or less complex.

This work examines a distributed and decentralized multi-agent solution in a simulated environment
for the 2D SLAM problem. For this, it uses the Sparse Extended Information Filter, along with other algorithms for navigation,
data association and map representation. The advantages of the distributed solution of the
SLAM problem, in comparison with the original problem, are workload sharing
between the agents and the information redundancy.

The developed solution has significantly less memory usage,
presenting a reduction of more than 50\% when compared to the more classic EKF-SLAM approach.
In addition, despite the lack of coordination between the agents
gains in exploration time were observed between the scenarios with two agents and with a single agent.

However, precisely because of the lack of
coordination between them, the time performance in the scenario with
three agents was lower than with a single one. Another limitation occurs
when the agents' initial poses are not informed, as the method used to 
estimate the relative pose between the agents did not prove to be robust
enough to ensure a successful map exchange whenever the robots get close 
of each other.