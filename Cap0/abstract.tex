The Simultaneous Localization and Mapping problem, known by the acronym SLAM, asks whether it is possible for a robot to be placed in an environment
unknown a priori, and incrementally build a map of this
environment while simultaneously locating itself on this map without the
need for a location infrastructure such as GPS.
Solving the SLAM problem is critical for mobile robotics
autonomy. However, although already solved, it is not a trivial task,
both from theoretical and implementation points of view.
Depending on robot dynamics, sensors used, computing power available or need for navigation and guidance, the solution
becomes more or less complex.We examine a distributed and decentralized multi-agent solution in a simulated environment
for the 2D SLAM problem. For such purpose, the solution uses the Sparse Extended Information Filter, along with other algorithms for navigation,
data association and map representation. The advantages of the distributed solution of the
SLAM problem, in comparison with the original problem, are workload sharing
among the agents and information redundancy. Our goal is to simulate a 
group of mobile robots mapping the environment where they were placed, 
in an active and decentralized fashion, while taking into account 
processing and memory constraints.
Beyond implementing algorithms that enable robots to solve the 
Distributed and Decentralized Active SLAM problem, a software framework is needed to simulate the environment and the agents. We used the Robot Operating System (ROS) to provide a set of software libraries and tools relevant to robotics 
development, and also a communication layer used by the many system 
modules (mapping, navigation, vision) in information exchanging. The 
simulation of the environment, sensors and agents, the Gazebo simulator 
was used.
The developed solution has significantly less memory usage,
presenting a reduction of more than 50\% when compared to the classic EKF-SLAM approach.
In addition, despite the lack of coordination between the agents,
gains in exploration time were observed in the scenario with two agents with respect to a single agent.
