The Simultaneous Localization and Mapping problem, known by the acronym SLAM, asks whether it is possible for a robot to be placed in an environment
unknown a priori, and incrementally build a map of this
environment while simultaneously locating itself on this map without the
need for a location infrastructure such as GPS.

Solving the SLAM problem is critical for mobile robotics
autonomy. However, although already solved, it is not a trivial task.
both from theoretical and implementation points of view.
Depending on robot dynamics, sensors used, computing power available or need for navigation and guidance, the solution
becomes more or less complex.

We examine a distributed and decentralized multi-agent solution in a simulated environment
for the 2D SLAM problem. For such purpose, the solution uses the Sparse Extended Information Filter, along with other algorithms for navigation,
data association and map representation. The advantages of the distributed solution of the
SLAM problem, in comparison with the original problem, are workload sharing
between the agents and information redundancy.

The developed solution has significantly less memory usage,
presenting a reduction of more than 50\% when compared to the classic EKF-SLAM approach.
In addition, despite the lack of coordination between the agents
gains, in exploration time were observed in the scenario with two agents with respect to a single agent.

However, precisely because of the lack of
coordination between them, we observe a degraded time performance in the scenario with
three agents relative to just a single agent. Another limitation occurs
when the agents' initial poses are unknown, as the method estimating the relative pose between agents did not prove to be robust
enough to ensure a successful map exchange whenever agents are in close 
proximity of one another.