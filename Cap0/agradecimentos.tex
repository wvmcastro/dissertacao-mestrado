Primeiro gostaria de agradecer meus pais, Adriana e Emerson, por terem prezado minha educação 
desde os primeiros anos, pois seu apreço e esforços somados à minha vontade me 
fizeram chegar num lugar incomum para as pessoas de nossa realidade 
socioeconômica. Além disso, eles sempre são os 
primeiros a acreditar em mim em meus momentos de dúvida. Devo 
tudo a eles.

Agradeço ao ITA, pois o instituto figura desde cedo em meu imaginário e estudar aqui sempre foi um sonho. Quero agradecer também ao 
meu orientador, professor Jacques, por propor um tema de pesquisa tão rico e apaixonante. Também agradeço à CAPES pelo fundamental fomento na 
forma de bolsa de estudos.

Por fim, quero agradecer aos meus amigos tanto de graduação do LIA-FACOM da UFMS, quanto da pós-graduação do ITA. A amizade e incentivo de vocês tornam a 
vida mais leve e gostosa de ser vivida, principalmente nos dias difíceis 
de um mestrando. Também quero agradecer a minha namorada, Camila, por 
todo amor e carinho durante a última parte dessa jornada, e também por toda compreensão nos momentos que fui ausente.

Além disso, agradeço especificamente à Camila e ao Ivan por terem lido 
diversas versões desse texto sugerindo alterações pertinentes para 
facilitar seu entendimento e prazer de leitura.