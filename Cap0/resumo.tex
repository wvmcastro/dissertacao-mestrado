O problema de Localização e Mapeamento Simultâneos, conhecido pela sigla SLAM, pergunta se é possível para um robô ser colocado em um ambiente 
desconhecido a priori, e incrementalmente construir um mapa deste 
ambiente enquanto simultaneamente se localiza neste mapa sem a 
necessidade de infraestrutura de localização como GPS.

A solução do problema SLAM é fundamental para a robótica móvel 
autônoma. Entretanto, apesar de já solucionado, não é uma tarefa trivial 
tanto do ponto de vista teórico como do ponto de vista da implementação. 
Dependendo da dinâmica do robô, sensores utilizados, recurso 
computacional disponível, necessidade de navegação e guiamento, a solução 
pode se tornar mais ou menos complexa.

Este trabalho investiga uma solução multiagente distribuída e descentralizada em ambiente simulado 
para o problema SLAM 2D. Para isso emprega o uso do Filtro de Informação 
Esparso, juntamente com outros algoritmos de navegação, associação de 
dados e de representação de mapas. As vantagens da solução distribuída do 
problema SLAM, em relação ao problema original, são a divisão da carga 
de trabalho entre os agentes e a redundância de informação.

A solução desenvolvida apresenta uso de memória significativamente 
reduzido, mais de 50\%, em relação à abordagem clássica EKF-SLAM. 
Além disso, apesar de não possuir coordenação entre os agentes, 
observaram-se ganhos no tempo de exploração no cenário com dois agentes 
em comparação com um único agente.

Porém, justamente pela falta de 
coordenação entre os robôs, o desempenho de tempo no cenário com 
três agentes foi inferior ao com um único agente. Outra limitação ocorre 
quando a pose inicial dos agentes não é informada, pois o método de 
estimação da pose relativa entre os agentes não se mostrou robusto o 
suficiente para garantir a troca dos mapas com sucesso sempre que eles se 
aproximam.