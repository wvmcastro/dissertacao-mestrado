O problema de Localização e Mapeamento Simultâneos, conhecido pela sigla SLAM, pergunta se é possível para um robô ser colocado em um ambiente 
desconhecido a priori, e incrementalmente construir um mapa deste 
ambiente enquanto simultaneamente se localiza neste mapa sem a 
necessidade de infraestrutura de localização como GPS.
A solução do problema SLAM é fundamental para a robótica móvel 
autônoma. Entretanto, apesar de já solucionado, não é uma tarefa trivial 
tanto do ponto de vista teórico como do ponto de vista da implementação. 
Dependendo da dinâmica do robô, sensores utilizados, recurso 
computacional disponível, necessidade de navegação e guiamento, a solução 
pode se tornar mais ou menos complexa.
Este trabalho investiga uma solução multiagente distribuída e descentralizada em ambiente simulado 
para o problema SLAM 2D. Para isso emprega o uso do Filtro de Informação 
Estendido Esparso, juntamente com outros algoritmos de navegação, associação de 
dados e de representação de mapas. As vantagens da solução distribuída do 
problema SLAM, em relação ao problema original, são a divisão da carga 
de trabalho entre os agentes e a redundância de informação. O objetivo deste trabalho é simular um grupo de 
robôs que mapeiem o ambiente onde estão inseridos, de maneira ativa e 
descentralizada, considerando restrições de memória e processamento.
Portanto, além de produzir algoritmos que capacitem os robôs a resolverem o 
problema SLAM Ativo Descentralizado e Distribuído, é preciso criar uma 
infraestrutura de software na qual o ambiente e os agentes serão simulados. Para 
isso utilizou-se o Sistema Operacional de Robô, ROS do inglês \textit{Robot 
Operating System}, que provê um conjunto de bibliotecas e ferramentas pertinentes 
ao desenvolvimento em robótica, além de uma camada de comunicação 
utilizada pelos diferentes módulos do sistema (mapeamento, navegação, 
visão) para trocarem informações. Para a simulação do ambiente, sensores e agentes, utilizou-se o simulador Gazebo.
A solução desenvolvida apresenta uso de memória significativamente 
reduzido, mais de 50\%, em relação à abordagem clássica EKF-SLAM. 
Além disso, apesar de não possuir coordenação entre os agentes, 
observaram-se ganhos no tempo de exploração no cenário com dois agentes 
em comparação com um único agente.
