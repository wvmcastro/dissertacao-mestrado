% \import{..}{wellington-commands.tex}

% talves refinar essa intro com: "Mobile robotic systems are developing very rapidly; how- ever, real-world applications in GPS-denied environments require robust mapping and perception techniques to en- able mobile systems to autonomously navigate complex environments." - review of multiple robot slam

O problema de Mapeamento e Localização Simultâneos conhecido pela sigla SLAM por conta do termo em inglês \textit{Simultaneous Localization and Mapping}, pergunta se é possível para um robô móvel ser colocado em um ambiente desconhecido a priori e incrementalmente construir um mapa deste ambiente enquanto simultaneamente se localiza neste mapa. Ou seja, tanto a trajetória da plataforma móvel quanto a localização das características do mapa (também conhecidas por \textit{landmarks}) são estimadas em tempo real sem a necessidade de nenhum conhecimento a priori de suas localizações \cite{durrant2006simultaneous}, ou infra estrutura de localização prévia, como GPS.

Além disso, SLAM também já foi conhecido como Mapeamento e Localização Concorrentes (CML, do inglês \textit{Concurrent Mapping and Localization}), porém este termo caiu em desuso a partir de 1995 quando o termo SLAM foi cunhando em \cite{durrant1996localization} no Simpósio Internacional de Pesquisa em Robótica, ISSR, onde originalmente era chamado \textit{Simultaneous Localization and Map Building}. A solução do problema de SLAM é fundamental para atingir a robótica móvel autônoma e independente de operadores \cite{durrant2006simultaneous}. Entretanto resolver o problema de localização e mapeamento simultâneos, apesar de solucionado, não é uma tarefa trivial tanto do ponto de vista teórico como do ponto de vista da implementação \cite{durrant1996localization}.

\subsection*{Caracterização do problema}
Imagine um robô dotado de um sensor extrínseco, capaz de capturar medidas relacionadas ao 
ambiente e, um sensor intrínseco capaz de medir os comandos de 
controle executados, se deslocando. Até o instante \emph{t} as seguintes 
quantidades são observadas:

\begin{itemize}
  \item $\bsubvec{x}{t}$: o vetor de estados descrevendo a pose do robô no instante \emph{t}
  \item $\bsubvec{x}{1:t} = \{ \bsubvec{x}{1}, \bsubvec{x}{2}, \dots, 
  \bsubvec{x}{t-1}, \bsubvec{x}{t} \}$: histórico de poses do robô até o instante \emph{t}
  \item $\bsubvec{u}{t}$: o vetor de controle executado pelo robô no instante \emph{t}
  \item $\bsubvec{u}{0:t} = \{ \bsubvec{u}{0}, \bsubvec{u}{1}, \dots, 
  \bsubvec{u}{t-1}, \bsubvec{u}{t} \}$: histórico de controles executados pelo robô até o instante \emph{t}
  \item $\bsubvec{z}{t}$: o vetor de medidas no instante \emph{t}
  \item $\bsubvec{z}{0:t} = \{ \bsubvec{z}{0}, \bsubvec{z}{1}, \dots, 
  \bsubvec{z}{t-1}, \bsubvec{z}{t} \}$: o conjunto de todas as medidas realizadas até o instante \emph{t}
  \item $\bvec{m}$: o vetor de mapa, constituído pelas posições das características do ambiente consideradas pelo robô
\end{itemize}

De maneira bastante sucinta, os problemas de SLAM consistem em estimar uma das 
seguintes distribuições de probabilidade:
\begin{gather}
  \pr{\bsubvec{x}{t}, \bvec{m} \given \bsubvec{z}{1:t} , \bsubvec{u}{0:t}, 
    \bsubvec{x}{0}}
  \label{eq:online-slam-probability} \\
  \pr{\bsubvec{x}{1:t}, \bvec{m} \given \bsubvec{z}{1:t} , \bsubvec{u}{1:t}, 
    \bsubvec{x}{0}}
  \label{eq:full-slam-probability}
\end{gather}

Além da solução da primeira distribuição se preocupar apenas em estimar o 
estado atual, enquanto na segunda toda a trajetória, o histórico de 
poses até o instante \emph{t}, é estimada. A diferença fundamental entre as 
duas soluções é que para calcular a primeira distribuição, não são utilizadas entradas de controle e medidas posteriores a um instante 
\emph{t} para estimar a pose $\bsubvec{x}{t}$. Enquanto na 
solução da segunda, medidas e controles posteriores podem ser utilizados para 
calcular poses anteriores a eles. Habitualmente os problemas de estimação 
dessas duas distribuições são conhecidos como \textit{online} SLAM e 
\textit{full} SLAM, respectivamente. 

Embora as definições do problema em \ref{eq:online-slam-probability} e 
\ref{eq:full-slam-probability} sejam simples, resolvê-lo está longe de ser. 
A depender das características do sistema como: a dinâmica do robô, sensores utilizados, recursos computacionais disponíveis, restrição de tempo real, e, 
necessidade de navegação e guiamento autônomos, sua solução pode se tornar mais 
ou menos complexa. Difícil também, é aprender terminologia utilizada, pelos 
pesquisadores, para cada uma dessas características. Parte da terminologia, 
pertinente a este trabalho, do problema de SLAM é abordada a seguir.

\subsection*{Taxonomia do problema SLAM}
Como é de se esperar, há termos específicos para tratar cada aspecto de um 
sistema SLAM. Esta Seção visa apresentar os termos pertinentes a este trabalho, 
a fim de estabelecer um vocabulário comum que será utilizado em todas as seções 
e capítulos subsequentes a este.

Como mencionado anteriormente, há duas classes de problemas de SLAM: 
\textit{online} SLAM e \textit{full} SLAM, e, portanto, duas classes de 
algoritmos para resolve-los. Os algoritmos \textit{full} SLAM também são chamados 
de \emph{offline SLAM} por serem normalmente utilizados em etapas de pós 
processamento, como refinamento de mapas. Esses algoritmos exigem mais 
recursos, tanto de processamento quanto de memória para serem processados. 
Portanto, há grande dificuldade para utiliza-los embarcados nos agentes durante 
a etapa de exploração do ambiente.

Em contra partida, os algoritmos que resolvem o problema de \textit{online} 
SLAM são comumente utilizados de maneira embarcada, pois tendem a 
consumir menos recursos computacionais. A pose e o mapa estimado por eles podem 
ser utilizados no processo de tomada de decisão do agente durante a execução da 
tarefa de mapeamento. 

Contudo, para que um robô consiga estimar precisamente sua pose $\bvec{x}$, é 
necessário alimentar os algoritmos com as medidas $\bvec{u}$ provenientes dos 
sensores intrínsecos (\textit{encoders}, giroscópios e acelerômetros), e, 
também, medidas $\bvec{z}$ do ambiente obtidas por sensores extrínsecos. O erro 
entre a posição esperada pelo robô de um objeto, dada a estimativa que o robô tem da sua pose e, a posição desse objeto lida pelo sensor extrínseco, pode ser 
utilizado para atualizar a confiança que o robô tem sobre a sua pose, por 
exemplo. Aqui, objeto significa qualquer aspecto do ambiente com características 
suficientes que permita-o ser identificado, podendo ser desde objetos
propriamente ditos como móveis e árvores, a pontos e quinas.

Essas medidas relacionadas ao ambiente, lidas pelos sensores extrínsecos 
(sonares, câmeras RGB, scanner laser, entre outros), possuem, em geral, duas 
componentes comumente denominados \textit{Range} e \textit{Bearing}. 
A componente \textit{Range} é a distância do sensor até o objeto medido. 
Enquanto \textit{Bearing} é a posição angular do objeto em relação ao sensor. 
Porém, nem todo sensor é capaz de fornecer essas duas medidas, câmeras RGB por 
exemplo, conseguem  informar apenas o \textit{Bearing}.

Então, de acordo com a presença/ausência dessas componentes os termos: 
\textit{Range Only}, \textit{Bearing Only} e \textit{Range-Bearing} SLAM são 
utilizados para identificar qual classe de medidas do ambiente está sendo
utilizada na solução do problema. \textit{Range Only} significa que a medida 
possui apenas a 
componente de distância. Em medidas \textit{Bearing Only} apenas a posição 
angular é lida. E em \textit{Range-Bearing} ambas as quantidades são lidas, em 
sistemas Range-Bearing SLAM são comumente utilizados sensores do tipo 
\emph{LIDAR} (\textit{Light Detection and Ranging}), que retornam uma nuvem de 
pontos onde cada ponto é descrito pela distância e posição angular em relação ao 
sensor.

Com um algoritmo capaz de estimar \ref{eq:online-slam-probability} ou 
\ref{eq:full-slam-probability} e sensores apropriados para alimenta-lo, um robô 
um robô é capaz de performar SLAM como foi apresentado até agora. Porém, ao mapear 
o ambiente, o robô pode explora-lo de maneira autônoma, ou, quando 
o cenário permite, ser controlado remotamente por um operador. Em cenários 
como a exploração de Marte tal controle é inviável. Quando a 
solução para o problema de SLAM também incorpora a geração de trajetórias, para 
que a exploração seja feita de forma autônoma (ativa), é denominada SLAM Ativo.

Até o momento, o problema de SLAM foi tratado como se a tarefa fosse resolvida 
por um único agente/robô. Porém, é possível integrar mais robôs para executarem 
a tarefa de maneira conjunta, surgindo assim uma série de benefícios. O primeiro, 
e mais óbvio, benefício é que a tarefa pode ser executada mais rápido já que a 
carga de trabalho é dividida entre os agentes. Outro ponto, é que mesmo que um 
agente venha a sofrer um dano, a tarefa ainda pode ser concluída, pois o sistema 
pode reagir e redistribuir a tarefa entre os robôs restantes. Porém, esses 
benefícios vêm com o preço de um sistema complexo que lida com a 
coordenação e cooperação dos robôs \cite{saeedi2016multiple}. Essa abordagem com 
múltiplos robôs é chamada de SLAM Distribuído.

Além disso, dependendo da arquitetura do fluxo de informação entre os agentes, 
a abordagem SLAM Distribuído é subdividida em Centralizada e Descentralizada \cite[p.~1316]{cadena2016past}. Na arquitetura centralizada, há um nó central 
responsável por processar e distribuir o mapa global composto 
pelo mapa local de cada agente do sistema, há portanto um único ponto de falha 
catastrófica, o nó central. Nessa arquitetura é geralmente mais simples manter 
consistência e consenso sobre o mapa global. 

Em contra partida, a arquitetura descentralizada não possui figura central, a 
comunicação e troca de mapas é realizada par a par entre os agentes. Neste arranjo 
todo o processamento é feito na ponta, consenso e convergência se tornam mais 
complicados, porém, o sistema se torna mais robusto com redundância de 
informação e ausência de falha catastrófica de um nó central.

% \subsection*{Soluções}
% Desde a proposição do problema diversas soluções foram propostas.

\section{Objetivo}
O objetivo deste trabalho é criar um sistema que consiste em um grupo de 
robôs capazes de mapear o ambiente onde estão inseridos, sem nenhuma infraestrutura de localização como GPS, de maneira ativa e 
descentralizada, preocupando-se com restrições de memória e processamento em 
ambiente simulado.

Portanto, além de produzir algoritmos que capacitem os robôs a resolverem o 
problema de SLAM Ativo Descentralizado e Distribuído, é preciso criar uma 
infraestrutura de software onde o ambiente e os agentes serão simulados. Para 
isso utilizou-se o Sistema Operacional de Robô, ROS do inglês \textit{Robot 
Operating System}, que é um \textit{framework} de código aberto e linguagem 
neutra \cite{quigley2009ros}, amplamente utilizado pela indústria e pela 
academia. Pois ele provê um conjunto de bibliotecas e ferramentas pertinentes 
ao cenário de desenvolvimento em robótica, além de uma camada de comunicação 
comum utilizada pelos diferentes módulos do sistema (mapeamento, navegação, 
visão) trocarem informações.

Dessa forma, ao utilizar o ROS este trabalho se torna facilmente reutilizável 
em outras pesquisas, permitindo que cada um de seus módulos 
(simulação, visualização, SLAM e navegação) possa ser explorado e até 
modificado de forma individual. Além disso, permite que mais módulos sejam 
adicionados, estendendo as capacidades do sistema aqui desenvolvido.

Para a simulação do ambiente, sensores e agentes utilizou-se o simulador Gazebo \cite{Koenig-2004-394}

tocar no assunto que o gazebo realiza simulacoes fidedignas de sensores, mass, 
friction, and numerous other physics variables
dizer que ele oferece um controle muito grande sobre quase todos os aspectos 
da simulacao desde condicoes de luz até coeficientes de atrito
textura, transparencia e cor

\section{Motivação}

\section{Organização do trabalho}