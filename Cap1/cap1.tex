O problema de Mapeamento e Localização Simultâneos conhecido pela sigla SLAM por conta do termo em inglês \textit{Simultaneous Localization and Mapping}, pergunta se é possível para um robô móvel ser colocado em um ambiente desconhecido a priori e incrementalmente construir um mapa deste ambiente enquanto simultaneamente se localiza neste mapa. Tanto a trajetória da plataforma móvel quanto a localização das características do mapa (também conhecidas por \textit{landmarks}) são estimadas em tempo real sem a necessidade de nenhum conhecimento a priori de suas localizações \cite{durrant2006simultaneous}.

Este problema também era conhecido como Mapeamento e Localização Concorrentes (CML, do inglês \textit{Concurrent Mapping and Localization}), porém este termo caiu em desuso a partir de 1995 quando o termo SLAM foi cunhando em \cite{durrant1996localization} no Simpósio Internacional de Pesquisa em Robótica, ISSR, onde originalmente era chamado \textit{Simultaneous Localization and Map Building}. A solução do problema de SLAM é fundamental para atingir a robótica móvel autônoma e independente de operadores \cite{durrant2006simultaneous}. Resolver o problema de localização e mapeamento simultâneos apesar de solucionado, não é uma tarefa trivial tanto do ponto de vista matemático como do ponto de vista da implementação \cite{durrant1996localization}.

\subsection*{Taxonomia do problema SLAM}
Vários termos são utilizados para especificar características de sistemas SLAM como tipo de sensor, formato da solução, passividade dos agentes, entre outras. E também estes termos não são únicos, podendo uma mesma característica ser referenciada por terminologias diferentes em diferentes fontes, por isso alguns termos serão apresentados e definidos para facilitar o entendimento da literatura e deste trabalho.

Há duas classes de algoritmos de SLAM, elas são caracterizadas pela quantidade de informação disponível em suas soluções. A primeira classe, chamada de \textit{online SLAM}, possui como solução o estado 


\subsection*{Soluções}
Desde a proposição do problema diversas soluções foram propostas.

\section{Objetivo}

\section{Motivação}

\section{Organização do trabalho}
