Well, the book is on the table. This work presents a control methodologie for the position of the  passive joints of an underactuated manipulator in a suboptimal way. The term underactuated refers to the fact that not all the joints or degrees of freedom of the system are equipped with actuators, which occurs in practice due to failures or as design result. The passive joints of manipulators like this are indirectly controlled by the motion of the active joints using the dynamic coupling characteristics. The utilization of actuation redundancy of the active joints allows the minimization of some criteria, like energy consumption, for example. Although the kinematic structure of an underactuated manipulator is identical to that of a similar fully actuated one, in general their dynamic characteristics are different due to the presence of passive joints. Thus, we present the dynamic modelling of an underactuated manipulator and the concept of coulpling index. This index is used in the sequence of the optimal control of the manipulator.
