\section{Resultados de mapeamento bem sucedido com agente único}
\label{app:single-agent-data}
As tabelas \ref{tab:single-agent-experiment-tab1} e \ref{tab:single-agent-experiment-tab2} registram a posição inicial, tempo de 
mapeamento de 90\% da área total do ambiente representado na Figura \ref{fig:environment}, o erro integral absoluto (IAE) médio da posição e da 
orientação para cada uma das simulações nas quais o robô conseguiu 
cumprir a tarefa com sucesso. Como o IAE aumenta conforme maior o tempo 
de simulação e como esse tempo é variável entre as diferentes 
realizações, calculou-se o IAE médio obtido pela razão do IAE pela quantidade 
de passos de simulação executados naquela realização.

\begin{table}[h]
\caption[Resultados mapeamento com agente único]{Resultados de mapeamento de cenário com um único agente}
\label{tab:single-agent-experiment-tab1}
\center
\begin{tabularx}{\textwidth}{@{}YYYYY@{}}
\hline
\# Realização & Posição inicial (m) & Tempo mapeamento 90\% da área (s) & IAE médio de posição (m) & IAE médio de orientação (rad)\\ \hline
1 & (2.7, -1.4)&       397& 0.019&         0.026 \\
2 & (1.1, 2.9)&       456& 0.044&         0.120 \\
3 & (3.0, -4.1)&       482& 0.034&         0.043 \\
4 & (1.3, 3.1)&       371& 0.022&         0.020 \\
5 & (-2.0, 2.1)&       345& 0.064&         0.103 \\
6 & (1.0, -1.1)&       435& 0.042&         0.081 \\
7 & (-0.9, 4.1)&       410& 0.041&         0.058 \\
8 & (-3.1, 3.6)&       445& 0.049&         0.126 \\
9 & (3.9, -1.0)&       441& 0.055&         0.102 \\
10 &(-3.3, 0.6)&       498& 0.058&         0.107 \\
11 &(-4.0, -0.9)&       353& 0.064&         0.075 \\
12 &(0.9, 2.9)&       370& 0.042&         0.059 \\
13 &(1.6, 2.7)&       546& 0.057&         0.065 \\
14 &(-4.5, -1.1)&       399& 0.026&         0.026 \\
15 &(3.2, -0.1)&       412& 0.034&         0.054 \\
16 &(0.2, 1.3)&       493& 0.059&         0.097 \\
17 &(-3.1, -0.1)&       376& 0.026&         0.027 \\
18 &(-3.2, 3.9)&       355& 0.052&         0.060 \\
\hline
\end{tabularx}
\end{table}

\begin{table}[]
\caption[Resultados mapeamento com agente único (Continuação)]{Resultados de mapeamento de cenário com um único agente (Continuação)}
\label{tab:single-agent-experiment-tab2}
\center
\begin{tabularx}{\textwidth}{@{}YYYYY@{}}
\hline
\# Realização & Posição inicial (m) & Tempo mapeamento 90\% da área (s) & IAE médio de posição (m) & IAE médio de orientação (rad)\\ \hline
19 &(4.4, 1.3)&       486& 0.086&         0.101 \\
20 &(2.8, 0.2)&       410& 0.014&         0.016 \\
21 &(3.1, 4.3)&       357& 0.009&         0.009 \\
22 &(0.0, 1.7)&       322& 0.015&         0.016 \\
23 &(3.8, 1.2)&       485& 0.010&         0.009 \\
24 &(3.2, -0.2)&       415& 0.013&         0.013 \\
25 &(-1.9, -0.9)&       408& 0.015&         0.011 \\
26 &(0.1, 1.0)&       396& 0.008&         0.008 \\
27 &(-1.3, 4.3)&       469& 0.011&         0.009 \\
28 &(-2.5, 4.0)&       440& 0.109&         0.048 \\
29 &(-2.4, 4.5)&       497& 0.070&         0.087 \\
30 &(-1.5, 4.1)&       482& 0.083&         0.076 \\
31 &(3.7, 1.5)&       431& 0.012&         0.011 \\
32 &(4.5, 3.0)&       387& 0.024&         0.007 \\
33 &(4.4, 1.6)&       544& 0.020&         0.022 \\
34 &(3.6, -1.3)&       302& 0.019&         0.012 \\
35 &(1.4, 3.0)&       470& 0.021&         0.010 \\
36 &(-1.8, -1.2)&       474& 0.010&         0.011 \\
37 &(4.2, 1.4)&       443& 0.110&         0.028 \\
38 &(3.8, 0.8)&       405& 0.007&         0.008 \\
39 &(-2.8, 4.4)&       494& 0.024&         0.011 \\
40 &(3.3, -1.1)&       416& 0.008&         0.011 \\
41 &(2.9, -1.4)&       330& 0.024&         0.019 \\
42 &(-2.0, 3.9)&       471& 0.086&         0.024 \\
43 &(2.7, -4.5)&       482& 0.028&         0.023 \\
44 &(-2.9, 2.9)&       484& 0.096&         0.036 \\
45 &(2.8, -1.7)&       434& 0.018&         0.017 \\
46 &(-0.3, -3.2)&       427& 0.021&         0.012 \\
47 &(0.8, 0.8)&       551& 0.011&         0.008 \\
48 &(-3.9, -3.1)&       351& 0.008&         0.011 \\
49 &(4.5, 1.3)&       403& 0.034&         0.010 \\
50 &(-2.7, 3.8)&       387& 0.013&         0.007 \\
51 &(-1.9, -1.1)&       371& 0.009&         0.009 \\
52 &(1.3, 0.9)&       403& 0.010&         0.012 \\
53 &(-3.3, -2.5)&       531& 0.012&         0.008 \\
54 &(1.4, 2.7)&       392& 0.019&         0.013 \\
55 &(3.2, -1.6)&       401& 0.013&         0.007 \\
56 &(-2.8, 4.5)&       413& 0.010&         0.010 \\
57 &(1.0, -1.0)&       410& 0.013&         0.008 \\
58 &(1.0, -4.4)&       437& 0.012&         0.015 \\
59 &(-2.9, -2.3)&       535& 0.019&         0.017 \\
60 &(0.3, 0.8)&       416& 0.013&         0.010 
\\ \hline
\end{tabularx}
  
\end{table}

\clearpage

\section{Resultados de mapeamento bem sucedido com dois agentes}
\label{app:two-agent-data}

As Tabelas \ref{tab:two-agent-experiment-pos-and-time-tab1} e \ref{tab:two-agent-experiment-pos-and-time-tab2} registram as posições 
iniciais e os tempos de mapeamento de 90\% da área do ambiente 
representado na Figura \ref{fig:environment} para o cenário com dois 
agentes com pose inicial conhecida. Enquanto as Tabelas \ref{tab:two-agent-experiment-iae-tab1} e \ref{tab:two-agent-experiment-iae-tab2} apresentam os dados de erro absoluto 
integral médio da posição e da orientação.

\begin{table}[]
\caption[Posição inicial e tempo de mapeamento para o cenário com dois agentes]{Posição inicial e tempo de mapeamento para o cenário com dois agentes}
\label{tab:two-agent-experiment-pos-and-time-tab1}
\center
\begin{tabularx}{\textwidth}{@{}YYYYY@{}}
\hline \\
\multirow{2}{*}{Realização} & \multicolumn{2}{c}{Posição inicial (m)} & \multicolumn{2}{c}{\begin{tabular}[c]{@{}c@{}}Tempo mapeamento \\ de 90\% da área (s)\end{tabular}} \\ \cline{2-5} 
 & Robô 1 & Robô 2 & Robô 1 & Robô 2 \\ \hline
1& (-1.7, 3.9)& (2.2, 1.9)&346&373 \\
2& (-4.2, -1.1)& (1.9, 1.5)&409&409 \\
3& (3.3, -1.0)& (-2.9, -2.1)&346&334 \\
4& (-1.8, 4.0)& (-2.8, 4.1)&317&373 \\
5& (-1.8, 4.3)& (-1.1, 0.0)&284&300 \\
6& (-4.5, -2.8)& (-1.6, 4.2)&289&345 \\
7& (2.9, -1.7)& (-1.7, 3.9)&281&244 \\
8& (1.0, 3.2)& (3.0, -4.2)&338&365 \\
9& (-2.5, 4.2)& (4.5, 1.3)&252&253 \\
10& (3.1, -1.5)& (2.1, 1.7)&312&272 \\
12& (3.7, -0.8)& (0.9, -4.5)&280&285 \\
13& (0.9, -1.2)& (1.5, 0.9)&354&280 \\
14& (0.4, 1.2)& (3.4, -0.6)&276&273 \\
15& (1.9, -2.9)& (-2.9, -2.1)&220&221 \\
16& (-0.8, 3.7)& (-2.7, 3.5)&279&290 \\
17& (1.9, -3.2)& (-1.2, -0.1)&307&324 \\
18& (-4.3, -3.0)& (-0.1, 0.9)&237&237 \\
19& (-1.0, 4.5)& (-3.0, 0.3)&262&262 \\
20& (2.0, 2.5)& (-2.6, 3.8)&316&339 \\
\hline
\end{tabularx}
\end{table}

\begin{table}[]
\caption[Posição inicial e tempo de mapeamento para o cenário com dois agentes (continuação)]{Posição inicial e tempo de mapeamento para o cenário com dois agentes (continuação)}
\label{tab:two-agent-experiment-pos-and-time-tab2}
\center
\begin{tabularx}{\textwidth}{@{}YYYYY@{}}
\hline \\
\multirow{2}{*}{Realização} & \multicolumn{2}{c}{Posição inicial (m)} & \multicolumn{2}{c}{\begin{tabular}[c]{@{}c@{}}Tempo mapeamento \\ de 90\% da área (s)\end{tabular}} \\ \cline{2-5} 
 & Robô 1 & Robô 2 & Robô 1 & Robô 2 \\ \hline
21& (0.0, 0.8)& (-2.7, 3.7)&375&399 \\
22& (-3.2, 3.9)& (1.1, 0.9)&447&452 \\
23& (3.2, -1.9)& (-3.0, 3.0)&394&253 \\
24& (4.1, 1.5)& (1.8, -3.2)&310&311 \\
25& (3.2, -3.8)& (1.1, -4.0)&473&362 \\
26& (-1.3, 4.5)& (-0.4, 0.2)&276&255 \\
27& (-0.8, 0.0)& (-4.0, 2.1)&244&242 \\
28& (1.7, 2.6)& (0.9, 3.5)&349&407 \\
29& (2.1, 2.0)& (1.1, 3.8)&336&319 \\
30& (1.8, -2.9)& (1.1, 3.4)&311&280 \\
31& (1.1, -4.2)& (2.2, 1.7)&358&378 \\
32& (-3.9, -3.0)& (2.9, -2.2)&347&475 \\
33& (1.9, -2.8)& (2.9, -1.1)&287&295 \\
34& (1.1, 1.2)& (1.9, 1.6)&308&298 \\
35& (-0.1, 0.7)& (-1.6, 4.2)&217&246 \\
36& (2.8, 4.1)& (3.0, -1.2)&317&407 \\
37& (-3.2, -2.0)& (-0.1, 0.5)&463&318 \\
38& (3.9, -0.9)& (1.5, 3.3)&316&296 \\
39& (1.1, -4.4)& (3.0, -1.3)&422&385 \\
40& (2.0, 1.1)& (-0.7, -0.1)&296&295 \\
41& (0.0, 1.7)& (-0.3, 0.0)&283&284 \\
42& (0.6, 1.0)& (2.9, -4.4)&354&358 \\
43& (-0.4, -0.1)& (-2.9, 3.1)&316&317 \\
44& (-3.2, 3.3)& (-4.5, 2.2)&474&429 \\
45& (2.8, -1.6)& (2.9, -4.1)&295&316 \\
46& (4.5, 3.1)& (-1.5, 4.5)&250&250\\
47\footnotemark{}& (1.9, -3.0)& (-2.1, 4.1) & 962 & 962
\\ \hline
\end{tabularx}
\end{table}


\begin{table}[]
\caption[IAE médio da posição e da orientação para o cenário com dois agentes (continuação)]{Erro absoluto integral médio da posição e da orientação para o cenário com dois agentes}
\label{tab:two-agent-experiment-iae-tab1}
\center
\begin{tabularx}{\textwidth}{@{}YYYYY@{}}
\hline \\
\multirow{2}{*}{Realização} & \multicolumn{2}{c}{IAE médio de posição (m)} & \multicolumn{2}{c}{IAE médio de orientação (rad)} \\ \cline{2-5} 
 & Robô 1 & Robô 2 & Robô 1 & Robô 2 \\ \hline
1&0.018&0.02&0.034&0.012 \\
2&0.026&0.02&0.037&0.031 \\
3&0.024&0.024&0.035&0.026 \\
4&0.06&0.06&0.04&0.038 \\
5&0.024&0.016&0.031&0.022 \\
6&0.028&0.032&0.059&0.093 \\
7&0.024&0.029&0.046&0.029 \\
8&0.02&0.019&0.022&0.023 \\
9&0.02&0.005&0.007&0.006 \\
10&0.039&0.03&0.018&0.047 \\
12&0.025&0.018&0.011&0.023 \\
13& nan&0.113&0.021&0.116 \\
14&0.006&0.01&0.006&0.006 \\
15&0.024&0.014&0.041&0.014 \\
16&0.036&0.042&0.032&0.051 \\
17&0.057&0.012&0.054&0.016 \\
18&0.011&0.059&0.013&0.064 \\
19&0.017&0.012&0.014&0.018 \\
20&0.019&0.03&0.022&0.023 \\
\\ \hline
\end{tabularx}
\end{table}


\begin{table}[]
\caption[IAE médio da posição e da orientação para o cenário com dois agentes (continuação)]{Erro absoluto integral médio da posição e da orientação para o cenário com dois agentes (continuação)}
\label{tab:two-agent-experiment-iae-tab2}
\center
\begin{tabularx}{\textwidth}{@{}YYYYY@{}}
\hline \\
\multirow{2}{*}{Realização} & \multicolumn{2}{c}{IAE médio de posição (m)} & \multicolumn{2}{c}{IAE médio de orientação (rad)} \\ \cline{2-5} 
 & Robô 1 & Robô 2 & Robô 1 & Robô 2 \\ \hline
21&0.011&0.007&0.009&0.008 \\
22&0.028&0.013&0.018&0.022 \\
23&0.013&0.013&0.027&0.011 \\
24&0.01&0.01&0.008&0.008 \\
25&0.157&0.036&0.053&0.017 \\
26&0.035&0.01&0.01&0.008 \\
27&0.012&0.01&0.014&0.009 \\
28&0.087&0.116&0.069&0.075 \\
29&0.025&0.026&0.054&0.043 \\
30&0.031&0.02&0.065&0.043 \\
31&0.05&0.029&0.126&0.02 \\
32&0.021&0.018&0.041&0.03 \\
33&0.014&0.009&0.009&0.01 \\
34&0.046&0.027&0.023&0.016 \\
35&0.047&0.017&0.037&0.015 \\
36&0.02&0.011&0.021&0.019 \\
37&0.038&0.062&0.101&0.08 \\
38&0.042&0.017&0.018&0.023 \\
39&0.012&0.027&0.029&0.023 \\
40&0.006&0.006&0.006&0.01 \\
41&0.006&0.008&0.005&0.007 \\
42&0.016&0.019&0.036&0.024 \\
43&0.01&0.01&0.01&0.01 \\
44&0.032&0.09&0.018&0.02 \\
45&0.023&0.046&0.032&0.027 \\
46&0.024&0.013&0.007&0.005\\
$47^1$ & 0.031 & 0.187 & 0.026 & 0.085
\\ \hline
\end{tabularx}
\end{table}

\footnotetext{\textit{outlier}}