\section{Resultados de mapeamento bem-sucedido com agente único}
\label{app:single-agent-data}
As tabelas \ref{tab:single-agent-experiment-tab1} e \ref{tab:single-agent-experiment-tab2} registram a posição inicial, tempo de 
mapeamento de 90\% da área total do ambiente representado na Figura \ref{fig:environment}, a integral do erro absoluto (IAE) de posição para cada uma das simulações nas quais o robô conseguiu 
cumprir a tarefa com sucesso. 

\begin{table}[h]
\caption[Resultados mapeamento com agente único]{Resultados de mapeamento de cenário com um único agente}
\label{tab:single-agent-experiment-tab1}
\center
\begin{tabularx}{\textwidth}{@{}YYYY@{}}
\hline
\# Realização & Posição inicial (m) & Tempo mapeamento 90\% da área (s) & IAE de posição (m) \\ \hline
1& (2.7, -1.4)&397&38.7\\
2& (1.1, 2.9)&456&102.77\\
3& (3.0, -4.1)&482&82.6\\
4& (1.3, 3.1)&371&42.4\\
5& (-2.0, 2.1)&345&114.2\\
6& (1.0, -1.1)&435&94.5\\
7& (-0.9, 4.1)&410&85.7\\
8& (-3.1, 3.6)&445&112.5\\
9& (3.9, -1.0)&441&123.3\\
10& (-3.3, 0.6)&498&148.6\\
11& (-4.0, -0.9)&353&118\\
12& (0.9, 2.9)&370&79.8\\
13& (1.6, 2.7)&546&158.2\\
14& (-4.5, -1.1)&399&53.3\\
15& (3.2, -0.1)&412&72.7\\
16& (0.2, 1.3)&493&148\\
17& (-3.1, -0.1)&376&50.9\\
18& (-3.2, 3.9)&355&93.8\\
\hline
\end{tabularx}
\end{table}

\begin{table}[]
\caption[Resultados mapeamento com agente único (Continuação)]{Resultados de mapeamento de cenário com um único agente (Continuação)}
\label{tab:single-agent-experiment-tab2}
\center
\begin{tabularx}{\textwidth}{@{}YYYYY@{}}
\hline
\# Realização & Posição inicial (m) & Tempo mapeamento 90\% da área (s) & IAE de posição (m) \\ \hline
19& (4.4, 1.3)&486&213.6\\
20& (2.8, 0.2)&410&29.9\\
21& (3.1, 4.3)&357&16.5\\
22& (0.0, 1.7)&322&24.3\\
23& (3.8, 1.2)&485&25.6\\
24& (3.2, -0.2)&415&28.3\\
25& (-1.9, -0.9)&408&32.2\\
26& (0.1, 1.0)&396&16.8\\
27& (-1.3, 4.3)&469&25.2\\
28& (-2.5, 4.0)&440&246.3\\
29& (-2.4, 4.5)&497&177.7\\
30& (-1.5, 4.1)&482&204.9\\
31& (3.7, 1.5)&431&27.8\\
32& (4.5, 3.0)&387&48\\
33& (4.4, 1.6)&544&56.9\\
34& (3.6, -1.3)&302&29.5\\
35& (1.4, 3.0)&470&51.2\\
36& (-1.8, -1.2)&474&25.4\\
37& (4.2, 1.4)&443&251.5\\
38& (3.8, 0.8)&405&15.2\\
39& (-2.8, 4.4)&494&61.9\\
40& (3.3, -1.1)&416&16.6\\
41& (2.9, -1.4)&330&41.5\\
42& (-2.0, 3.9)&471&207.8\\
43& (2.7, -4.5)&482&69.1\\
44& (-2.9, 2.9)&484&238.3\\
45& (2.8, -1.7)&434&40.8\\
46& (-0.3, -3.2)&427&45.7\\
47& (0.8, 0.8)&551&31.6\\
48& (-3.9, -3.1)&351&13.8\\
49& (4.5, 1.3)&403&70.3\\
50& (-2.7, 3.8)&387&25.8\\
51& (-1.9, -1.1)&371&17.7\\
52& (1.3, 0.9)&403&21.5\\
53& (-3.3, -2.5)&531&31.4\\
54& (1.4, 2.7)&392&38.1\\
55& (3.2, -1.6)&401&26.3\\
56& (-2.8, 4.5)&413&21.7\\
57& (1.0, -1.0)&410&27.2\\
58& (1.0, -4.4)&437&26.4\\
59& (-2.9, -2.3)&535&51.8\\
60& (0.3, 0.8)&416&27.9\\
\hline
\end{tabularx}
  
\end{table}

\clearpage

\section{Resultados de mapeamento bem-sucedido com dois agentes}
\label{app:two-agent-data}

As Tabelas \ref{tab:two-agent-experiment-pos-and-time-tab1} e \ref{tab:two-agent-experiment-pos-and-time-tab2} registram as posições 
iniciais e os tempos de mapeamento de 90\% da área do ambiente 
representado na Figura \ref{fig:environment} para o cenário com dois 
agentes com pose inicial conhecida. Enquanto as Tabelas \ref{tab:two-agent-experiment-iae-tab1} e \ref{tab:two-agent-experiment-iae-tab2} apresentam os dados da integral do erro absoluto de posição.

\begin{table}[]
\caption[Posição inicial e tempo de mapeamento para o cenário com dois agentes]{Posição inicial e tempo de mapeamento para o cenário com dois agentes}
\label{tab:two-agent-experiment-pos-and-time-tab1}
\center
\begin{tabularx}{\textwidth}{@{}YYYYY@{}}
\hline \\
\multirow{2}{*}{Realização} & \multicolumn{2}{c}{Posição inicial (m)} & \multicolumn{2}{c}{\begin{tabular}[c]{@{}c@{}}Tempo mapeamento \\ de 90\% da área (s)\end{tabular}} \\ \cline{2-5} 
 & Robô 1 & Robô 2 & Robô 1 & Robô 2 \\ \hline
1& (-1.7, 3.9)& (2.2, 1.9)&346&373 \\
2& (-4.2, -1.1)& (1.9, 1.5)&409&409 \\
3& (3.3, -1.0)& (-2.9, -2.1)&346&334 \\
4& (-1.8, 4.0)& (-2.8, 4.1)&317&373 \\
5& (-1.8, 4.3)& (-1.1, 0.0)&284&300 \\
6& (-4.5, -2.8)& (-1.6, 4.2)&289&345 \\
7& (2.9, -1.7)& (-1.7, 3.9)&281&244 \\
8& (1.0, 3.2)& (3.0, -4.2)&338&365 \\
9& (-2.5, 4.2)& (4.5, 1.3)&252&253 \\
10& (3.1, -1.5)& (2.1, 1.7)&312&272 \\
12& (3.7, -0.8)& (0.9, -4.5)&280&285 \\
13& (0.9, -1.2)& (1.5, 0.9)&354&280 \\
14& (0.4, 1.2)& (3.4, -0.6)&276&273 \\
15& (1.9, -2.9)& (-2.9, -2.1)&220&221 \\
16& (-0.8, 3.7)& (-2.7, 3.5)&279&290 \\
17& (1.9, -3.2)& (-1.2, -0.1)&307&324 \\
18& (-4.3, -3.0)& (-0.1, 0.9)&237&237 \\
19& (-1.0, 4.5)& (-3.0, 0.3)&262&262 \\
20& (2.0, 2.5)& (-2.6, 3.8)&316&339 \\
\hline
\end{tabularx}
\end{table}

\begin{table}[]
\caption[Posição inicial e tempo de mapeamento para o cenário com dois agentes (continuação)]{Posição inicial e tempo de mapeamento para o cenário com dois agentes (continuação)}
\label{tab:two-agent-experiment-pos-and-time-tab2}
\center
\begin{tabularx}{\textwidth}{@{}YYYYY@{}}
\hline \\
\multirow{2}{*}{Realização} & \multicolumn{2}{c}{Posição inicial (m)} & \multicolumn{2}{c}{\begin{tabular}[c]{@{}c@{}}Tempo mapeamento \\ de 90\% da área (s)\end{tabular}} \\ \cline{2-5} 
 & Robô 1 & Robô 2 & Robô 1 & Robô 2 \\ \hline
21& (0.0, 0.8)& (-2.7, 3.7)&375&399 \\
22& (-3.2, 3.9)& (1.1, 0.9)&447&452 \\
23& (3.2, -1.9)& (-3.0, 3.0)&394&253 \\
24& (4.1, 1.5)& (1.8, -3.2)&310&311 \\
25& (3.2, -3.8)& (1.1, -4.0)&473&362 \\
26& (-1.3, 4.5)& (-0.4, 0.2)&276&255 \\
27& (-0.8, 0.0)& (-4.0, 2.1)&244&242 \\
28& (1.7, 2.6)& (0.9, 3.5)&349&407 \\
29& (2.1, 2.0)& (1.1, 3.8)&336&319 \\
30& (1.8, -2.9)& (1.1, 3.4)&311&280 \\
31& (1.1, -4.2)& (2.2, 1.7)&358&378 \\
32& (-3.9, -3.0)& (2.9, -2.2)&347&475 \\
33& (1.9, -2.8)& (2.9, -1.1)&287&295 \\
34& (1.1, 1.2)& (1.9, 1.6)&308&298 \\
35& (-0.1, 0.7)& (-1.6, 4.2)&217&246 \\
36& (2.8, 4.1)& (3.0, -1.2)&317&407 \\
37& (-3.2, -2.0)& (-0.1, 0.5)&463&318 \\
38& (3.9, -0.9)& (1.5, 3.3)&316&296 \\
39& (1.1, -4.4)& (3.0, -1.3)&422&385 \\
40& (2.0, 1.1)& (-0.7, -0.1)&296&295 \\
41& (0.0, 1.7)& (-0.3, 0.0)&283&284 \\
42& (0.6, 1.0)& (2.9, -4.4)&354&358 \\
43& (-0.4, -0.1)& (-2.9, 3.1)&316&317 \\
44& (-3.2, 3.3)& (-4.5, 2.2)&474&429 \\
45& (2.8, -1.6)& (2.9, -4.1)&295&316 \\
46& (4.5, 3.1)& (-1.5, 4.5)&250&250\\
47\footnotemark{}& (1.9, -3.0)& (-2.1, 4.1) & 962 & 962
\\ \hline
\end{tabularx}
\end{table}


\begin{table}[]
\caption[IAE de posição para o cenário com dois agentes (continuação)]{Integral do Erro Absoluto de posição para o cenário com dois agentes}
\label{tab:two-agent-experiment-iae-tab1}
\center
\begin{tabularx}{.7\textwidth}{@{}YYY@{}}
\hline \\
\multirow{2}{*}{Simulação} & \multicolumn{2}{c}{IAE de posição (m)} \\ \cline{2-3} 
 & Robô 1 & Robô 2 \\ \hline\\
1&34.4&37.6\\
2&54.8&42.6\\
3&43.4&42.9\\
4&114.4&115.5\\
5&36.6&24\\
6&49.5&56.8\\
7&35&42.9\\
8&37.3&36.1\\
9&25.7&7.1\\
10&62.4&48.4\\
12&36.7&26.9\\
13& nan&206.1\\
14&8.3&14.6\\
15&27.6&16.2\\
16&53.9&62.1\\
17&95.3&19.6\\
18&14.3&72.3\\
19&22.8&16.7\\
20&32.3&52.3\\
\hline
\end{tabularx}
\end{table}


\begin{table}[]
\caption[IAE de posição para o cenário com dois agentes (continuação)]{Integral do Erro Absoluto de posição para o cenário com dois agentes (continuação)}
\label{tab:two-agent-experiment-iae-tab2}
\center
\begin{tabularx}{.7\textwidth}{@{}YYY@{}}
\hline \\
\multirow{2}{*}{Realização} & \multicolumn{2}{c}{IAE de posição (m)} \\ \cline{2-3} 
 & Robô 1 & Robô 2 \\ \hline
21&22.4&15\\
22&72.4&43.4\\
23&43.5&45.4\\
24&183&241\\
25&65.3&31\\
26&61.1&21.5\\
27&27.2&27\\
28&15.3&15.2\\
29&381.8&86.6\\
30&49.6&14.9\\
31&15.3&12.9\\
32&97.3&56.2\\
33&50.7&43\\
34&21.8&13\\
35&41.8&24.1\\
36&91.4&147\\
37&69.7&27.9\\
38&26.5&58.8\\
39&8.6&9.2\\
40&9.2&11.9\\
41&30&35.5\\
42&15.8&17.2\\
43&50.1&32.5\\
44&77.2&218.5\\
45&36.9&74.9\\
46&30.8&17.1\\
$47^1$&149&909.8\\
\hline
\end{tabularx}
\end{table}

\footnotetext{\textit{outlier}}