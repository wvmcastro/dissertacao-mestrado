%%% Exemplo de utilização da classe ITA
%%%
%%%   por        Fábio Fagundes Silveira   -  ffs [at] ita [dot] br
%%%              Benedito C. O. Maciel     -  bcmaciel [at] ita [dot] br
%%%              Giovani Volnei Meinertz   -  giovani [at] ita [dot] br
%%%    	         Hudson Alberto Bode       -  bode [at] ita [dot]br
%%%    	         P. I. Braga de Queiroz    -  pi [at] ita [dot] br
%%%    	         Jorge A. B. Gripp         -  gripp [at] ita [dot] br
%%%    	         Juliano Monte-Mor         -  jamontemor [at] yahoo [dot] com [dot] br
%%%    	         Tarcisio A. B. Gripp      -  tarcisio.gripp [at] gmail [dot] com
%%%    	         
%%%   Versão para overleaf:
%%%   por           Alejandro A. Rios Cruz - aarc.88@gmail.com 	         
%%%                 Saulo Gómez            - sagomezs@unal.edu.co 
%%%  IMPORTANTE: O texto contido neste exemplo nao significa absolutamente nada.  :-)
%%%              O intuito aqui eh demonstrar os comandos criados na classe e suas
%%%              respectivas utilizacoes.
%%%
%%%  Tese.tex  2016-08-25
%%%  $HeadURL: http://www.apgita.org.br/apgita/teses-e-latex.php $
%%%
%%% ITALUS
%%% Instituto Tecnológico de Aeronáutica --- ITA, Sao Jose dos Campos, Brasil
%%%                   http://groups.yahoo.com/group/italus/
%%% Discussion list: italus {at} yahoogroups.com
%%%
%++++++++++++++++++++++++++++++++++++++++++++++++++++++++++++++++++++++++++++++
% Para alterar o TIPO DE DOCUMENTO, preencher a linha abaixo \documentclass[?]{?}
%   \documentclass[tg]{ita}			= Trabalho de Graduacao
%   \documentclass[tgfem]{ita}	= Para Engenheiras
%   								msc     		= Dissertacao de Mestrado
%   								mscfem   		= Para Mestras
%   								dsc      		= Tese de Doutorado
%   								dscfem   		= Para Doutoras
%   								quali    		= Exame de Qualificacao
%   								qualifem 		= Exame de Qualificacao para Doutoras
% Para 'Draft Version'/'Versao Preliminar' com data no rodape, adicionar 'dv':
%   \documentclass[dsc, dv]{ita} 
% Para trabalhos em Inglês, adicionar 'eng':
%   \documentclass[dsc, eng]{ita}
%		\documentclass[dsc, eng, dv]{ita}
%++++++++++++++++++++++++++++++++++++++++++++++++++++++++++++++++++++++++++++++
\documentclass[msc, dv]{ita}
% Quando alterar a classe, por exemplo de [msc] para [msc, eng]) rode mais uma vez o botão BUILD OUTPUT caso haja erro
\usepackage{ae}
\usepackage{graphicx}
\usepackage{epsfig}
\usepackage{amsmath}
\usepackage{amssymb} 
\usepackage{subfig}
\usepackage{multirow}
\usepackage{float}
\usepackage{amsthm}
\usepackage{url}         % formats URL addresses properly
\usepackage{appendix}    % allows appendix section to be included
\usepackage{lscape}      % allows a page to be rendered in landscape mode
\usepackage{multicol}    % allows text in multi columns
\usepackage{cancel}      % needed to show canceled terms in equations
\usepackage{lettrine}
\usepackage{float}
\usepackage{placeins}


%HHHHHHHHHHHHHHHHHHHHHHHHHHHHHHHHHHHHHHHHHHHHHHHHHHHHHHHHHHHHHHHHHHHHHHHHHHHHHHHHHHHHHHHHHHHHHHHHHHHHHHHHHHHH
%\usepackage{subfigure}
%\usepackage{subfigmat}
%PACOTEFIGURAS_SE _ERRADO_ESXCLUIR_ACIMA
\usepackage{booktabs}
%PACOTETABELAS_SE _ERRADO_ESXCLUIR_ACIMA
%HHHHHHHHHHHHHHHHHHHHHHHHHHHHHHHHHHHHHHHHHHHHHHHHHHHHHHHHHHHHHHHHHHHHHHHHHHHHHHHHHHHHHHHHHHHHHHHHHHHHHHHHHHHH

\usepackage{wellington-packages}
% \usepackage{wellington-commands}

%++++++++++++++++++++++++++++++++++++++++++++++++++++++++++++++++++++++++++++++
% Espaçamento padrão de todo o documento
%++++++++++++++++++++++++++++++++++++++++++++++++++++++++++++++++++++++++++++++
\onehalfspacing

%singlespacing Para um espaçamento simples
%onehalfspacing Para um espaçamento de 1,5
%doublespacing Para um espaçamento duplo

%++++++++++++++++++++++++++++++++++++++++++++++++++++++++++++++++++++++++++++++
% Identificacoes (se o trabalho for em inglês, insira os dados em inglês)
% Para entradas abreviadas de Professora (Profa.) em português escreva: Prof$^\textnormal{a}$.
%++++++++++++++++++++++++++++++++++++++++++++++++++++++++++++++++++++++++++++++
\course{Engenharia Eletrônica e Computação}  % Programa de PG ou Curso de Graduação
\area{Sistemas e Controle} % Área de concentração na PG (Não utilizado no caso de TG)

% Autor do trabalho: Nome Sobrenome
\authorgender{masc}                     %sexo: masc ou fem
\author{Wellington Vieira Martins}{de Castro}
\itaauthoraddress{Av. Engenheiro Francisco José Longo, 633. Apartamento 113}{12.245-906}{São José dos Campos--SP}

% Titulo da Tese/Dissertação
\title{SLAM distribuído envolvendo navegação, guiamento e fusão sensorial para reconstrução 2D}

% Orientador
\advisorgender{masc}                    % masc ou fem
\advisor{Prof.~Dr.}{Jacques Waldmann}{ITA}

% Coorientador (Caso não haja coorientador, colocar ambas as variáveis \coadvisorgender e \coadvisor comentadas, com um % na frente)
% \coadvisorgender{fem}									% masc ou fem
% \coadvisor{Prof$^\textnormal{a}$.~Dr$^\textnormal{a}$.}{Doralice Serra}{OVNI}

% Pró-reitor da Pós-graduação
\bossgender{masc}												% masc ou fem
\boss{Prof.~Dr.}{John von Neumann}

%Coordenador do curso no caso de TG
% \bosscoursegender{masc}									% masc ou fem
% \bosscourse{Prof.~Dr.}{John Walker}

% Palavras-Chaves informadas pela Biblioteca -> utilizada na CIP
\kwcip{Cupim}
\kwcip{Dilema}
\kwcip{Construção}

% membros da banca examinadora

\examiner{Prof. Dr.}{Alan Turing}{Presidente}{ITA}
\examiner{Prof. Dr.}{Linus Torwald}{}{UXXX}
\examiner{Prof. Dr.}{Richard Stallman}{}{UYYY}
\examiner{Prof. Dr.}{Donald Duck}{}{DYSNEY}
\examiner{Prof. Dr.}{Mickey Mouse}{}{DISNEY}

% Data da defesa (mês em maiúsculo, se trabalho em inglês, e minúsculo se trabalho em português) 
\date{5}{março}{2015}

% Número CDU - (somente para TG)
% \cdu{621.38}

% Glossario
\makeglossary
\frontmatter

\begin{document}
% Folha de Rosto e Capa para o caso do TG
\maketitle

% Dedicatoria: Nao esqueca essa secao  ... :-)
\begin{itadedication}
Aos amigos da Graduação e Pós-Graduação do ITA por motivarem tanto a criação deste template pelo Fábio Fagundes Silveira quanto por motivarem a mim e outras pessoas a atualizarem e aprimorarem este excelente trabalho.
\end{itadedication}

% Agradecimentos
\begin{itathanks}
Primeiramente, gostaria de agradecer ao Dr. Donald E. Knuth, por ter desenvolvido o \TeX.

Ao Dr. Leslie Lamport, por ter criado o \LaTeX, facilitando muito a utilização do \TeX, e assim, eu não ter que usar o Word.

Ao Prof. Dr. Meu Orientador, pela orientação e confiança depositada na realização deste trabalho.

Ao Dr. Nelson D'Ávilla, por emprestar seu nome a essa importante via de trânsito na cidade de São José dos Campos.

Ah, já estava esquecendo... agradeço também, mais uma vez ao \TeX, por ele não possuir vírus de macro :-)

\end{itathanks}

% Epígrafe
\thispagestyle{empty}
\ifhyperref\pdfbookmark[0]{\nameepigraphe}{epigrafe}\fi
\begin{flushright}
\begin{spacing}{1}
\mbox{}\vfill
{\sffamily\itshape
``If I have seen farther than others,\\
it is because I stood on the shoulders of giants.''\\}
--- \textsc{Sir~Isaac Newton}
\end{spacing}
\end{flushright}

% Resumo
\begin{abstract}
\noindent
O problema de Localização e Mapeamento Simultâneos, conhecido pela sigla SLAM, pergunta se é possível para um robô ser colocado em um ambiente 
desconhecido a priori, e incrementalmente construir um mapa deste 
ambiente enquanto simultaneamente se localiza neste mapa sem a 
necessidade de infraestrutura de localização como GPS.

A solução do problema SLAM é fundamental para a robótica móvel 
autônoma. Entretanto, apesar de já solucionado, não é uma tarefa trivial 
tanto do ponto de vista teórico como do ponto de vista da implementação. 
Dependendo da dinâmica do robô, sensores utilizados, recurso 
computacional disponível, necessidade de navegação e guiamento, a solução 
pode se tornar mais ou menos complexa.

Este trabalho investiga uma solução multiagente descentralizada e distribuída em ambiente simulado 
para o problema SLAM 2D. Para isso emprega o uso do Filtro de Informação 
Esparso, juntamente com outros algoritmos de navegação, associação de 
dados e de representação de mapas. As vantagens da solução distribuída do 
problema SLAM, em relação ao problema original, são a divisão da carga 
de trabalho entre os agentes e a redundância de informação.

A solução desenvolvida apresenta uso de memória significativamente 
reduzido, mais de 50\%, em relação à abordagem mais clássica EKF-SLAM. 
Além disso, apesar de não possuir coordenação entre os agentes 
observaram-se ganhos no tempo de exploração no cenário com dois agentes 
em comparação com um único agente.

Porém, justamente pela falta de 
coordenação entre os robôs, o desempenho de tempo no cenário com 
três agentes foi inferior ao com um único agente. Outra limitação ocorre 
quando a pose inicial dos agentes não é informada, pois o método de 
estimação da pose relativa entre os agentes não se mostrou robusto o 
suficiente para garantir a troca dos mapas com sucesso sempre que eles se 
aproximam.
\end{abstract}

% Abstract
\begin{englishabstract}
\noindent
The Simultaneous Localization and Mapping problem, known by the acronym SLAM, asks whether it is possible for a robot to be placed in an environment
unknown a priori, and incrementally build a map of this
environment while simultaneously locating itself on this map without the
need for a location infrastructure such as GPS.

Solving the SLAM problem is critical for mobile robotics
autonomy. However, although already solved, it is not a trivial task.
both from theoretical and implementation points of view.
Depending on robot dynamics, sensors used, computing power available or need for navigation and guidance, the solution
becomes more or less complex.

We examine a distributed and decentralized multi-agent solution in a simulated environment
for the 2D SLAM problem. For such purpose, the solution uses the Sparse Extended Information Filter, along with other algorithms for navigation,
data association and map representation. The advantages of the distributed solution of the
SLAM problem, in comparison with the original problem, are workload sharing
between the agents and information redundancy.

The developed solution has significantly less memory usage,
presenting a reduction of more than 50\% when compared to the classic EKF-SLAM approach.
In addition, despite the lack of coordination between the agents
gains, in exploration time were observed in the scenario with two agents with respect to a single agent.

However, precisely because of the lack of
coordination between them, we observe a degraded time performance in the scenario with
three agents relative to just a single agent. Another limitation occurs
when the agents' initial poses are unknown, as the method estimating the relative pose between agents did not prove to be robust
enough to ensure a successful map exchange whenever agents are in close 
proximity of one another.
\end{englishabstract}

\import{.}{wellington-commands.tex}

% Lista de figuras
\listoffigures %opcional

% Lista de tabelas
\listoftables %opcional

% Lista de abreviaturas
\listofabbreviations
\begin{longtable}{ll}
CML & \textit{Concurrent Mapping and Localization} (Mapeamento e Localização Concorrentes)\\
GPS & \textit{Global Positioning System} (Sistema de Posicionamento Global)\\
SLAM & \textit{Simultaneous Localization and Mapping} (Localização e Mapeamento Simultâneos) \\
ISSR & \textit{International Symposium on Robotics Research}\\
& (Simpósio Internacional de Pesquisa em Robótica)\\
ROS & \textit{Robot Operating System} (Sistema Operacional de Robô)\\
KF & \textit{Kalman Filter} (Filtro de Kalman)\\
EKF & \textit{Extended Kalman Filter} (Filtro de Kalman Estendido)\\
EIF & \textit{Extended Information Filter} (Filtro de Informação Estendido)\\
SEIF & \textit{Sparse Extended Information Filter} (Filtro de Informação Estendido Esparso)\\
IMU & \textit{Inertial Measurement Unit} (Unidade de Medição Inercial)\\
LiDAR & \textit{Light Detection and Ranging} \\
RANSAC & \textit{Random Sample Consensus} \\
IAE & \textit{Integral Absolute Error} (Erro Absoluto Integral)
\end{longtable}

 %opcional

% Lista de simbolos
\listofsymbol

\begin{longtable}{ll}
$a$ & Distância\\
$\textbf{a}$ & Vetor de distâncias\\
$\textbf{e}_{j}$ & Vetor unitário de dimensão $n$ e com o $j$-ésimo componente igual a $1$ \\
$\textbf{K}$ & Matriz de rigidez\\
$m_1$ & Massa do cumpim\\
$\delta_{k-k_f}$ & Delta de Kronecker no instante $k_f$\\

\end{longtable}

 %opcional

% Sumario
\tableofcontents

\mainmatter
% Os capitulos comecam aqui

\chapter{Introdução}
\import{Cap1}{cap1}

\chapter{Visão Geral do Sistema}
\import{Cap2}{cap2}

\chapter{Filtragem em SLAM}
\import{Cap3}{cap3}

\chapter{Conclusão}
\import{Cap4}{cap4}

% REFERENCIAS BIBLIOGRAFICAS
\renewcommand\bibname{\itareferencesnamebabel} %renomear título do capítulo referências
\bibliography{Referencias/referencias}

% Apendices
\appendix
\chapter{Tópicos de Dilema Linear} %opcional
\section{Algoritmo EKF-SLAM}
\label{app:alg-ekf-slam}
As três operações principais do EKF-SLAM (predição, atualização e inserção de novas \textit{landmarks}) estão descritas nos Algoritmos \ref{alg:ekf-slam-prediction}, \ref{alg:ekf-slam-update} e \ref{alg:ekf-slam-landmark-insertion}, respectivamente. Por fim, o Algoritmo \ref{alg:ekf-slam-full} descreve o EKF-SLAM completo, composto pelas três operações. A correção na orientação, para que ela permaneça dentro do 
intervalo $\brac{-\pi, \pi}$, é omitida tanto no Algoritmo \ref{alg:ekf-slam-prediction} quanto no \ref{alg:ekf-slam-update}.

\begin{algorithm}
  \setstretch{1.5}
  \caption{Etapa de predição do EKF-SLAM}
  \label{alg:ekf-slam-prediction}
\begin{algorithmic}[1]
\Procedure{EKF-SLAM-Predição}{$\mean{t-1}, \cov{t-1}, \bsubvec{u}{t}$}
  \State $\predmean{t} \gets \begin{bmatrix}
    \mb{g}_R(\mean{R,t-1}, \bsubvec{u}{t}) \\ \mean{M}
  \end{bmatrix}$
  \State $\predcov{RR,t} \gets \bsubvec{G}{R, t}\cov{RR, t-1} \bsubvecT{G}{R,t}$
  \State $\predcov{RM, t} \gets \bsubvec{G}{R, t} \cov{RM, t-1}$
  \State $\predcov{t} \gets \begin{bmatrix}
    \predcov{RR, t} & \predcov{RM, t} \\
    \predcov{RM, t}^T & \cov{MM}
  \end{bmatrix}$
  \State \Return $\predmean{t}, \predcov{t}$
\EndProcedure
\end{algorithmic}
\end{algorithm}

\begin{algorithm}[h]
  \setstretch{1.25}
  \caption{Etapa de atualização do EKF-SLAM}
  \label{alg:ekf-slam-update}
\begin{algorithmic}[1]
  \Procedure{EKF-SLAM-Atualização}{$\predmean{t}, \predcov{t}, \bvec{y}, j = \text{índice da \textit{landmark}}$}
  \State $\bvec{z} \gets \bvec{y} - \mb{h}^j(\predmean{t})$
  \State $\bvec{Z} \gets \begin{bmatrix}
    \bsubvec{H}{R}^j & \bsubvec{H}{M}^j
  \end{bmatrix}
  \begin{bmatrix}
    \predcov{RR} & \predcov{RM_j} \\
    \predcov{RM_j}^T & \predcov{M_j M_j}
  \end{bmatrix}
  \begin{bmatrix}
    \bsubvec{H}{R}^j \\ \bsubvec{H}{M}^j
  \end{bmatrix} + \bsubvec{Q}{t}$
  \State $\bvec{K} \gets \begin{bmatrix} \cov{RR, t} & \cov{RM^j, t} \\ 
    \cov{MR, t} & \cov{MM^j, t} 
  \end{bmatrix}
  \begin{bmatrix}
    \brac{\bsubvec{H}{R}^j}^T \\ \brac{\bsubvec{H}{M}^j}^T
  \end{bmatrix} \bvec{Z}^{-1}$
  \State $\mean{t} \gets \bvec{K} \bvec{z}$
  \State $\cov{t} \gets \predcov{t} - \bvec{K}\bvec{Z}\bvecT{K}$
  \State \Return $\mean{t}, \cov{t}$
  \EndProcedure
\end{algorithmic}
\end{algorithm}

\begin{algorithm}
  \setstretch{1.5}
  \caption{Etapa de inserção de nova \textit{landmark} do EKF-SLAM}
  \label{alg:ekf-slam-landmark-insertion}
\begin{algorithmic}[1]
  \Procedure{EKF-SLAM-Inserção-Nova-Landmark}{$\predmean{t}, \predcov{t}, \bvec{y}, j$}
    \State $\predmean{t}^* \gets \begin{bmatrix}
      \predmean{t} \\ \mb{f}(\predmean{t}, \bvec{y})
    \end{bmatrix}$
    \State $\predcov{t}^* \gets \begin{bmatrix}
      \predcov{t} & \predcov{t} \bvec{F}_X^T\\
       \predcov{t} \bvec{F}_X &  \bvec{F}_X \predcov{t} \bvec{F}_X^T 
       + \bvec{F}_Y \bvec{Q} \bvec{F}_Y^T
    \end{bmatrix}$
    \State \Return $\predmean{t}^*, \predcov{t}^*$
  \EndProcedure
\end{algorithmic}
\end{algorithm}

\begin{algorithm}[h]
  \setstretch{1.25}
  \caption{EKF-SLAM}
  \label{alg:ekf-slam-full}
\begin{algorithmic}[1]
  \Procedure{EKF-SLAM}{$\mean{t-1}, \cov{t-1}, \bsubvec{u}{t}, \bvec{y}^{1:k}, \bvec{c}^{1:k}$} 
    \State $\predmean{t}, \predcov{t} \gets \textproc{EKF-SLAM-Predição}(\mean{t-1}, \cov{t-1}, \bsubvec{u}{t})$
    \State $\mathcal{L}^{*} \gets \{\}$ \Comment{Conjunto de novas landmarks}
    \For {$(\bvec{y}^i \in \bvec{y}^{1:k})$}
      \State $j \gets \bvec{c}^i$
      \If {landmark $j$ está no mapa $\bsubvec{x}{M}$}
        \State $\predmean{t}, \predcov{t} \gets \textproc{EKF-SLAM-Atualização}(\predmean{t}, \predcov{t}, \bvec{y}^i, j)$
      \Else
        \State $\mathcal{L}^* \gets \mathcal{L}^* + \{(j, \bvec{y}^i)\}$
      \EndIf
    \EndFor
    \For {$\mathcal{L}^i \in \mathcal{L}^*$}
      \State $j, \bvec{y}^i \gets \mathcal{L}^i$
      \State $\predmean{t}, \predcov{t} \gets \textproc{EKF-SLAM-Inserção-Nova-Landmark}(\predmean{t}, \predcov{t}, \bvec{y}^i, j)$
    \EndFor
    \State $\mean{t}, \cov{t} \gets \predmean{t}, \predcov{t}$
    \State \Return $\mean{t}, \cov{t}$
  \EndProcedure
\end{algorithmic}
\end{algorithm}

% Anexos
\annex
\chapter{Exemplo de um Primeiro Anexo} %opcional
\section{Lema da Inversão. Fórmula de  Sherman/Morrison}
\label{annex:inversion-lemma}
A fórmula de Sherman/Morrison, também conhecida como lema da inversão especializado,
é definido a seguir, retirado de \cite[p.~50]{bongard2006probabilistic}.

\begin{lem}
Para qualquer matrizez quadradas invertíveis $\bvec{R}$ e $\bvec{Q}$ e qualquer matriz $\bvec{P}$ com dimensões apropriadas, o seguinte é verdadeiro:
\begin{equation}
\parentheses{\bvec{R} + \bvec{PQ}\bvecT{P}}^\mathbf{{-1}} = \bvec{R^{-1}} - \bvec{R^{-1}}\bvec{P}\parentheses{\bvec{Q^{-1}} + \bvecT{P}\bvec{R^{-1}P}}^\mathbf{{-1}}\bvecT{P}\bvec{R^{-1}}
\end{equation}
assumindo que todas as matrizes acima podem ser invertidas como definido na premissa.
\begin{proof}
Defina $\bvec{\Psi} = \parentheses{\bvec{Q^{-1}} + \bvecT{P}\bvec{R^{-1}}\bvec{P}}^\mathbf{-1}$. É suficiente mostrar que:
\begin{equation*}
\parentheses{\bvec{R^{-1}} - \bvec{R^{-1}}\bvec{P}\bvec{\Psi}\bvecT{P}\bvec{R^{-1}}} \parentheses{\bvec{R} + \bvec{PQ}\bvecT{P}} = \bvec{I}
\end{equation*}
Isso é mostrado através de uma série de manipulações:
\begin{equation}
\begin{aligned}
    &= \blue{\bvec{R^{-1} R}} + \bvec{R^{-1} P Q P^T} - \bvec{R^{-1} P \Psi P^T \magenta{R^{-1} R}} - \bvec{R^{-1} P \Psi P^T R^{-1} P Q P^T} \\
    &= \blue{\bvec{I}} + \bvec{R^{-1} P Q P^T} - \bvec{R^{-1} P \Psi P^T \magenta{I}} - \bvec{R^{-1} P \Psi P^T R^{-1} P Q P^T}\\
    &= \bvec{I} + \bvec{\magenta{R^{-1} P} Q P^T} - \bvec{\magenta{R^{-1} P} \Psi P^T} - \bvec{\magenta{R^{-1} P} \Psi P^T R^{-1} P Q P^T}\\
    &= \bvec{I} + \bvec{R^{-1} P} \left[\bvec{Q P^T} - \bvec{\Psi P^T} - \bvec{\Psi P^T R^{-1} P Q P^T}\right]\\
    &= \bvec{I} + \bvec{R^{-1} P} \left[\bvec{Q P^T} - \bvec{\Psi \blue{Q^{-1} Q} P^T} - \bvec{\Psi P^T R^{-1} P Q P^T}\right]\\
    &= \bvec{I} + \bvec{R^{-1} P} \left[\bvec{Q P^T} - \bvec{\magenta{\Psi} Q^{-1}\blue{ Q P^T}} - \bvec{\magenta{\Psi} P^T R^{-1} P \blue{Q P^T}}\right]\\
    &= \bvec{I} + \bvec{R^{-1} P} \left[\bvec{Q P^T} - \bvec{\magenta{\Psi}} \left(\bvec{ Q^{-1}} - \bvec{P^T R^{-1} P}\right)\bvec{\blue{Q P^T}}\right]\\
    &= \bvec{I} + \bvec{R^{-1} P} \left[\bvec{Q P^T} - \bvec{\Psi} \blue{\left(\bvec{ Q^{-1}} - \bvec{P^T R^{-1} P}\right)}\bvec{Q P^T}\right]\\
    &= \bvec{I} + \bvec{R^{-1} P} \left[\bvec{Q P^T} - \bvec{\Psi} \blue{\bvec{\Psi^{-1}}}\bvec{Q P^T}\right]\\
    &= \bvec{I} + \bvec{R^{-1} P} \left[\bvec{Q P^T} - \bvec{Q P^T}\right]\\
    &= \bvec{I} 
\end{aligned}
\end{equation}

\end{proof}
\label{lem:inversion-lemma}

\end{lem}


\section{Inversão na forma de blocos}
\label{annex:blockwise-inversion-lemma}
Seja a matriz $(m+n) \times (m+n)$, $\bvec{M}$, particionada na seguinte forma de blocos:
\begin{equation}
    \bvec{M} = \begin{bmatrix}
        \bvec{A} & \bvec{B} \\ \bvec{C} & \bvec{D}
    \end{bmatrix}
\end{equation}
onde as matrizes $\mat{A}{m}{m}$ e $\mat{D}{n}{n}$ são invertíveis, então temos que $\bvec{M}^{-1}$ é dado por:
\begin{equation}
    \newcommand{\commonTerm}{\parentheses{\bvec{D-C}\bvec{A}^{-1}\bvec{B}}^{-1}}
    \bvec{M}^{-1} = \begin{bmatrix}
        \bvec{A}^{-1} + \bvec{A}^{-1}\bvec{B} \commonTerm \bvec{C} \bvec{A}^{-1} & -\bvec{A}^{-1}\bvec{B}\commonTerm \\
        -\commonTerm \bvec{C} \bvec{A}^{-1} & \commonTerm
    \end{bmatrix}
    \label{eq:blockwise-matrix-inversion}
\end{equation}

% Glossario
%\itaglossary
%\printglossary

% Folha de Registro do Documento
% Valores dos campos do formulario
\FRDitadata{25 de março de 2015}
\FRDitadocnro{DCTA/ITA/DM-018/2015} %(o número de registro você solicita a biblioteca)
\FRDitaorgaointerno{Instituto Tecnológico de Aeronáutica -- ITA}
%Exemplo no caso de pós-graduação: Instituto Tecnol{\'o}gico de Aeron{\'a}utica -- ITA
\FRDitapalavrasautor{Cupim; Cimento; Estruturas}
\FRDitapalavrasresult{Cupim; Dilema; Construção}
%Exemplo no caso de graduação (TG):
%\FRDitapalavraapresentacao{Trabalho de Graduação, ITA, São José dos Campos, 2015. \NumPenultimaPagina\ páginas.}
%Exemplo no caso de pós-graduação (msc, dsc):
\FRDitapalavraapresentacao{ITA, São José dos Campos. Curso de Mestrado. Programa de Pós-Graduação em Engenharia Aeronáutica e Mecânica. Área de Sistemas Aeroespaciais e Mecatrônica. Orientador: Prof.~Dr. Adalberto Santos Dupont. Coorientadora: Prof$^\textnormal{a}$.~Dr$^\textnormal{a}$. Doralice Serra. Defesa em 05/03/2015. Publicada em 25/03/2015.}
\FRDitaresumo{O problema de Localização e Mapeamento Simultâneos, conhecido pela sigla SLAM, pergunta se é possível para um robô ser colocado em um ambiente 
desconhecido a priori, e incrementalmente construir um mapa deste 
ambiente enquanto simultaneamente se localiza neste mapa sem a 
necessidade de infraestrutura de localização como GPS.

A solução do problema SLAM é fundamental para a robótica móvel 
autônoma. Entretanto, apesar de já solucionado, não é uma tarefa trivial 
tanto do ponto de vista teórico como do ponto de vista da implementação. 
Dependendo da dinâmica do robô, sensores utilizados, recurso 
computacional disponível, necessidade de navegação e guiamento, a solução 
pode se tornar mais ou menos complexa.

Este trabalho investiga uma solução multiagente descentralizada e distribuída em ambiente simulado 
para o problema SLAM 2D. Para isso emprega o uso do Filtro de Informação 
Esparso, juntamente com outros algoritmos de navegação, associação de 
dados e de representação de mapas. As vantagens da solução distribuída do 
problema SLAM, em relação ao problema original, são a divisão da carga 
de trabalho entre os agentes e a redundância de informação.

A solução desenvolvida apresenta uso de memória significativamente 
reduzido, mais de 50\%, em relação à abordagem mais clássica EKF-SLAM. 
Além disso, apesar de não possuir coordenação entre os agentes 
observaram-se ganhos no tempo de exploração no cenário com dois agentes 
em comparação com um único agente.

Porém, justamente pela falta de 
coordenação entre os robôs, o desempenho de tempo no cenário com 
três agentes foi inferior ao com um único agente. Outra limitação ocorre 
quando a pose inicial dos agentes não é informada, pois o método de 
estimação da pose relativa entre os agentes não se mostrou robusto o 
suficiente para garantir a troca dos mapas com sucesso sempre que eles se 
aproximam.}
%  Primeiro Parametro: Nacional ou Internacional -- N/I
%  Segundo parametro: Ostensivo, Reservado, Confidencial ou Secreto -- O/R/C/S
\FRDitaOpcoes{N}{O}
% Cria o formulario
\itaFRD

\end{document}
% Fim do Documento. O massacre acabou!!! :-)
