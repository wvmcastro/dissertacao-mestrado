\section{O ambiente}


\section{Modelo do robô}

\section{Medidas e o modelo de medida \textit{Range-Bearing}}
A medida gerada pelo sensor LiDar, embarcado no TurtleBot, consiste em uma nuvem de pontos planar. Essa nuvem de pontos é processada e dela são extraídas estimativas dos centros dos cilindros presentes no ambiente. Esses 
centros são as medidas utilizadas pelo algoritmo de SLAM, eles são descritos 
em termos de coordenadas polares $(r, \theta)$ no sistema de coordenadas do sensor.

\subsection{Modelo de medida \textit{Range-Bearing}}
O modelo de medida calcula a medida que espera-se ser lida pelo sensor, quanto o sistema está no estado $\bsubvec{x}{t}$. Na Figura \ref{fig:range-bearing-measurement-schematic}, é representado um sistema composto por um robô e três \textit{landmarks} i, j e k. Logo o vetor de estados, $\bvec{x}$, é formado pela pose do robô, e pelas posições das landmarks no mapa:
\begin{equation}
  \bvec{x} = \begin{bmatrix}
    \phi & x & y & m_x^i & m_y^i & m_x^j & m_y^j & m_x^k & m_y^k
  \end{bmatrix}^T
\end{equation}
o modelo de medida para a j-ésima \textit{landmark} é dado por
\renewcommand{\arraystretch}{1.5}
\begin{equation}
  \mb{h}^j(\bvec{x}) = \begin{bmatrix}
    r^j \\ \theta^j
  \end{bmatrix} = \begin{bmatrix}
    \sqrt{(m_x^j - x_l)^2 + (m_y^j - y_l)^2} \\
    \arctan\left(\dfrac{m_y^j - y_l}{m_x^j - x_l}\right) - \phi
  \end{bmatrix}
  \label{eq:range-bearing-measurement-model}
\end{equation}
\renewcommand{\arraystretch}{1}
onde
\begin{equation}
  \begin{cases}
    x_l = x + d \cos \phi\\
    y_l = y + d \sin \phi
  \end{cases}
\end{equation}

\begin{figure}[h]
  \centering
  \includestandalone[width=\textwidth]{figs/range-bearing}
  \caption{Esquemático do modelo de medida \textit{Range-Bearing}. O sistema de coordenadas do robô é representado em magenta, e o sistema de coordenadas do sensor em azul. A medida $(r^j, \theta^j)$ se refere à i-ésima \textit{landmark} no mapa.}
  \label{fig:range-bearing-measurement-schematic}
\end{figure}

Como o modelo de medida em \ref{eq:range-bearing-measurement-model} é não linear, é necessário lineariza-lo para utilizá-lo em soluções como EKF-SLAM, e seus derivados como SEIF-SLAM. Sua matriz jacobiana, $\bvec{H}$, para a j-ésima \textit{landmark} é descrita na Equação \ref{eq:measurement-model-jacobian-full}, 
ela é composta pelos jacobianos $\bsubvec{H}{R}$, calculado com relação à pose do robô, e pelo jacobiano $\bsubvec{H}{M}$, calculado com relação à 
posição da \textit{landmark} no vetor de estado.
\renewcommand{\arraystretch}{2}
\begin{equation}
  \bsubvec{H}{R}^j = \begin{bmatrix}
    \dfrac{d}{r^j}\parentheses{\delta_x \sin\phi - \delta_y \cos\phi} &  \dfrac{-\delta_x}{r^j} & \dfrac{-\delta_y}{r^j}\\
    -\parentheses{\dfrac{d}{\brac{r^j}^2}\parentheses{\delta_y \sin\phi + \delta_x \cos\phi} + 1} & \dfrac{\delta_y}{\brac{r^j}^2} & \dfrac{-\delta_x}{\brac{r^j}^2}
    \end{bmatrix}
  \label{eq:measurement-model-jacobian-robot-part}
\end{equation}
\renewcommand{\arraystretch}{2}
\begin{equation}
  \bsubvec{H}{M}^j = \begin{bmatrix}
      \dfrac{\delta_x}{r^j} & \dfrac{\delta_y}{r^j} \\
      \dfrac{-\delta_y}{\brac{r^j}^2} & \dfrac{\delta_x}{\brac{r^j}^2}
    \end{bmatrix}
  \label{eq:measurement-model-jacobian-map-part}
\end{equation}
\renewcommand{\arraystretch}{1}
\begin{equation}
  \bvec{H}^j(\bvec{x}) = \begin{bmatrix}
    \bsubvec{H}{R}^j & \bvec{0} & \cdots & \bsubvec{H}{M}^j & \cdots & \bvec{0}
  \end{bmatrix} 
  \label{eq:measurement-model-jacobian-full}
\end{equation}

\subsection{Modelo de medida inverso \textit{Range-Bearing}}
\label{sec:inverse-measurement-model}
O modelo de medida descrito na Seção anterior é também conhecido como modelo de medida direto, ele calcula a medida que espera-se ler quando o sistema está em um dado estado. Mas também há o modelo de medida inverso, que calcula um estado a partir de uma medida, esse modelo é útil durante o descobrimento de novas \textit{landmarks}, pois ele da meios para que suas posições sejam  incorporadas no vetor de estados, na Figura \ref{fig:range-bearing-inverse-model-schematic} está representado o momento no qual o robô descobre a p-ésima \textit{landmark} do ambiente.
\begin{figure}[h]
  \includestandalone[width=\textwidth]{figs/range-bearing-inverse}
  \caption{Esquemático do modelo de medida \textit{Range-Bearing}. O sistema de coordenadas do robô é representado em magenta, e o sistema de coordenadas do sensor em azul. Os círculos representam \textit{landmarks}, as conhecidas pelo robô (portanto presentes no vetor de estados), em cinza, e recém descobertas, em laranja. A p-ésima \textit{landmark} acaba de ser encontrada pelo robô.}
  \label{fig:range-bearing-inverse-model-schematic}
\end{figure}

O modelo de medida inverso $\mb{f}(\bullet, \bullet)$ é descrito na Equação \ref{eq:inverse-measurement-model}. Como pode ser observado, assim como o 
modelo de media ``direto'', o modelo de medida inverso é não linear, logo devemos lineariza-lo para utilizá-lo com o EKF-SLAM e seus 
algoritmos derivados. Sua matriz jacobiana, $\bvec{F}$, na Equação \ref{eq:inverse-measurement-model-jacobian-full} é composta pelos jacobianos 
parciais em relação ao vetor de estados, $\bvec{F}_X$, e à medida da nova 
landmark encontrada, $\bvec{F}_Y$, mostrados nas Equações \ref{eq:inverse-measurement-model-jacobian-state-part} e \ref{eq:inverse-measurement-model-jacobian-measurement-part}, respectivamente.

\begin{equation}
  \mb{f}(\bvec{x}, \bvec{y}^{p}) = 
    \begin{bmatrix}
        m_x^p \\
        m_y^p \\
    \end{bmatrix} =
    \begin{bmatrix}
        x_l + r^p \cos(\phi + \theta^p)\\
        y_l + r^p \sin(\phi + \theta^p) 
    \end{bmatrix}
    \label{eq:inverse-measurement-model}
\end{equation}

\begin{equation}
  \bvec{F}_X = 
    \begin{bmatrix}
      \begin{bmatrix}
          -d \sin{\phi} - r^p \sin(\phi + \theta^p)  & 1 & 0\\
          d\cos{\phi} + r^p\cos(\phi + \theta^p) & 0 & 1
      \end{bmatrix} & \nullmat{2}{n-3} \end{bmatrix}
  \label{eq:inverse-measurement-model-jacobian-state-part}
\end{equation}

\begin{equation}
  \bvec{F}_Y = \begin{bmatrix}
       \cos(\phi + \theta^p) & -r^p\sin(\phi + \theta^p)\\
       \sin(\phi + \theta^p) & r^p\cos(\phi + \theta^p)
    \end{bmatrix}   
  \label{eq:inverse-measurement-model-jacobian-measurement-part}
\end{equation}

\begin{equation}
  \bvec{F} = \begin{bmatrix}
    \bvec{F}_X & \bvec{F}_Y
  \end{bmatrix}
  \label{eq:inverse-measurement-model-jacobian-full}
\end{equation}

\section{Simulação}

\section{Visualização}
